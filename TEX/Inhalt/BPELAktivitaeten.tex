%!TEX root = ../Masterarbeit.tex
\section{Liste m�glicher Aktivit�ten der BPEL}
\label{sec:ListeBPELAktivitaeten}

Die folgenden Listen basieren auf \Zitat[Abschnitt~10~u.~11]{OASIS2007a}.

\small

\small
\begin{longtable}{|p{0.14\textwidth}|p{0.43\textwidth}|p{0.35\textwidth}|}
\caption{Ausgew�hlte elementare BPEL-Aktivit�ten} \\
\hline
\label{tab:ListeBPELAktivitaetenElementar}
\textbf{Aktivit�t} & \textbf{Beschreibung} & \textbf{Beispiel} \\
\hline
\XMLElement{invoke} & 
Aufruf einer Operation eines Webservice. Dabei wird zwischen \Fachbegriff{One-way}- und \Fachbegriff{Request-response}-Kommunikation unterschieden. Eventuelle Input- und Output-Nachrichten werden hierbei angegeben. \XMLElement{invoke}-Elemente k�nnen weitere Elemente (wie \zB \XMLElement{faultHandler} zur Fehlerbehandlung) beinhalten. & 
\vspace{-0.8cm}
\begin{verbatim}
<invoke 
  partnerLink="PLName"
  portType="PTName"
  operation="OName"
  inputVariable="VarName"
  outputVariable="VarName">
\end{verbatim}\\
\hline
\XMLElement{receive} & Empf�ngt eine Nachricht von einem Partner. Dazu muss die Operation angegeben werden, die der Prozess anbietet um die Nachricht entgegenzunehmen. Die Nachricht kann in einer \XMLElement{variable} gespeichert werden. & 
\vspace{-0.8cm}
\begin{verbatim}
<receive 
  partnerLink="PLName"
  portType="PTName"
  operation="OName"
  variable="VarName">
\end{verbatim}\\
\hline
\XMLElement{reply} & Antwortet auf die Nachricht eines Partners (nur sinnvoll bei \Fachbegriff{Request-Response}-Kommunikation). & 
\vspace{-0.8cm}
\begin{verbatim}
<reply 
  partnerLink="PLName"
  portType="PTName"
  operation="OName"
  variable="VarName">
\end{verbatim}\\
\hline
\XMLElement{assign} & Zuweisung von Werten zu Variablen. &
\vspace{-0.8cm}
\begin{verbatim}
<assign>
  <copy>
    <from>
      <literal>
        <![CDATA[Wert]]>
      </literal>
    </from>
    <to variable="myVar" />
 </copy>
</assign>
\end{verbatim} \\
\hline
\XMLElement{throw} & Signalisierung eines Fehlers (analog zu Programmiersprachen). & 
\vspace{-0.8cm}
\begin{verbatim}
<throw 
  faultName="FName" 
  faultVariable="VarName">
\end{verbatim} \\
\hline
\XMLElement{wait} & L�sst den Prozess eine gewisse Zeit lang warten. & 
\vspace{-0.8cm}
\begin{verbatim}
<wait>
  <until>
    '2002-12-24T18:00'
  </until>
</wait>
\end{verbatim} \\
\hline
\XMLElement{exit} & Beendet den Prozess sofort. & 
\vspace{-0.8cm}
\begin{verbatim}
<exit>
\end{verbatim} \\
\hline
\end{longtable}

\small
\begin{longtable}{|p{0.14\textwidth}|p{0.43\textwidth}|p{0.35\textwidth}|}
\caption{Ausgew�hlte strukturierte BPEL-Aktivit�ten} \\
\hline
\label{tab:ListeBPELAktivitaetenStrukturiert}
\textbf{Aktivit�t} & \textbf{Beschreibung} & \textbf{Beispiel} \\
\hline
\XMLElement{sequence} & Sequentielles Abarbeiten der angegebenen Aktivit�ten. & 
\vspace{-0.8cm}
\begin{verbatim}
<sequence>
  <invoke>...</invoke>
  <invoke>...</invoke>
</sequence>
\end{verbatim}\\
\hline
\XMLElement{flow} & Paralleles Abarbeiten der angegebenen Aktivit�ten. & 
\vspace{-0.8cm}
\begin{verbatim}
<flow>
  <invoke>...</invoke>
  <invoke>...</invoke>
</flow>
\end{verbatim}\\
\hline
\XMLElement{if} & Konditionale Abfragen (vergleichbar zur Programmierung). & 
\vspace{-0.8cm}
\begin{verbatim}
<if>
  <condition>
    ...
  </condition>
  <sequence>
    ...
  </sequence>
  <elseif>
    ...
  </elseif>
  <else>
    ...
  </else>
</if>
\end{verbatim}\\
\hline
\XMLElement{while} & Wiederholung der angegebenen Aktivit�ten (vergleichbar zur Programmierung). & 
\vspace{-0.8cm}
\begin{verbatim}
<while>
  <condition>
    $orderDetails > 100
  </condition>
  <scope>...</scope>
</while>
\end{verbatim}\\
\hline
\XMLElement{scope} & Ver�ndern des Kontextes in dem die angegebenen Aktivit�ten ablaufen. So k�nnen \zB neue Variablen deklariert oder eine andere Fehlerbehandlung definiert werden. Das \XMLElement{scope}-Element stellt eigentlich keine Aktivit�t dar, soll hier aber trotzdem erw�hnt werden. & 
\vspace{-0.8cm}
\begin{verbatim}
<scope>
  <faultHandlers>
    ...
  </faultHandlers>
  <flow>
    <invoke>...</invoke>
  </flow>
</scope>
\end{verbatim}\\
\hline
\end{longtable}

\normalsize
