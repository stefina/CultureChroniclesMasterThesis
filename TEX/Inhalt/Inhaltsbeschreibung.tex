%!TEX root = ../Masterarbeit.tex
%switch Inhaltsbeschreibung on
\newcommand*{\INHALTSBESCHREIBUNG}{}%
\newcommand{\InhaltGrundlagen}{
	Vorfeldrecherche
	Status Quo				(1-2)
	- was gibt es denn gerade so, um "sowas" im entferntesten zu machen? so zu suchen, etc.
	- projekte, die ihre eigenen ansätze von stöbern in der vergangenheit haben, werden kurz vorgestellt und sollen einerseitz den bedarf an solch einer app verdeutlichen und natürlich andererseits zeigen, wie sehr so ne app benötigt wird
	Feldanalyse				(3)
	- feldanalysen für die bereiche musik, film, geschichtliche Ereignisse und Bild/Kunst.
	- wenn ich "solche" daten haben will, welche möglichkeiten bieten sich mir, welche probleme treten auf (zb. wie bekomme ich musik aus den 90ern, heute wahrscheinlich durch sampler oder playlisten, die sind aber finit und selten bis gar nicht dynamisch)
	- soll zeigen: Ja, ich komme schon irgendwie an die Daten ran, aber besonders schick oder komfortabel ist das nicht
	Untersuchung vorhandener Suchdienste (3)
	- Was spucken die gängigen Suchmaschinen denn so aus, wenn ich gezielt suche? wie brauchbar ist das? wie verlässlich? wie konsistent?
Analyse
	Anforderungsanalyse		(8)
	- was will ich jetzt eigentlich? was soll die app können? was soll man machen können? also noch eher grob gehalten á la "nutzer soll suchergebnisse sehen und gleichzeitig musik und video konsumieren können" und "inhalte musik, film, news, wetter sind denkbar", und eingabemöglichkeiten, schön auf die spitze getrieben und so
	Feature-Liste			(5)}