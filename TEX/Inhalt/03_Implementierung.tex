%!TEX root = ../Masterarbeit.tex
\chapter{Implementierung}
\label{cha:implementierung}


\section{Quellen}
%Erforderliche und ben�tigte Quellen und APIs
\subsection{MusicBrainz}
MusicBrainz ist eine offene Musik-Enzyklop�die, die f�r Mensch und Maschine lesbare Informationen �ber Musik �ffentlich zug�nglich macht.\footcite[Vgl.][]{musicbrainz} Sie enth�lt umfangreiche Daten �ber K�nstler und deren ver�ffentlichte Musik, insbesondere auch Informationen �ber einzelne Erscheinungen. Dar�ber hinaus kann verf�gt MusicBrainz auf den K�nstler-Profilen s�mtliche Verlinkungen zu weiteren Referenzen im Internet wie Wikipedia-Artikel, Youtube-Channel, Facebook-Seite, Wikidata-Eintrag \uvm


\begin{itemize}
	\item Datenquellen
		\begin{itemize}
			\item MusicBrainz
			\item RateYourMusic
			\item OneMusicAPI
			\item Music Story Pro
			\item Discogs
			\item Wolfram Alpha
			\item IMDb
			\item themobiedb.org
			\item RottenTomatoes API
			\item wikipedia
		\end{itemize}
	\item Dienste
		\begin{itemize}
			\item Spotify
			\item youtube
		\end{itemize}
\end{itemize}

\subsection{Wikipedia-Dumps}
http://en.wikipedia.org/w/index.php?title=Special:Export

\section{Techniken}
\subsection{RESTful-API}
\subsection{Wrapper}
\subsection{Web-Scraping}
\subsection{Download/Dumps}


\section{Architektur der Web-Applikation}
Die Web-Applikation wird serverseitig mittels Javascript umgesetzt. Die Software-Plattform Node.js\footcite{nodejs} bietet ein einfaches Konzept zur Erstellung und Inbetriebnahme von skalierbaren Netzwerk-Applikationen. Vorteile im Bezug auf \arbeitstitel \ ist das event-driven Konzept von gro�em Vorteil, da die Kommunikation mit mehreren APIs um einiges beschleunigt werden kann, indem die API-Requests von der Asynchronit�t Gebrauch machen.

\subsection{Express.js}
Das Web-Application-Framework \textbf{Express.js}\footcite{expressjs} wird verwendet, da es eine Palette von Funktionalit�t zur einfachen Implementierung von Web-Applikationen mit \textbf{node.js} bietet und trotzdem flexibel und leichtgewichtig ist. Es ist als Schicht zwischen Funktionalit�t und Frontend zu verstehen und agiert als API.\footcite[Vgl.][]{expressjs}

\subsection{MongoDB und NoSQL}
MongoDB\footcite{mongodb} ist eine dokumentenorientierte Datenbank. Im Gegensatz zu relationalen Datenbanken l�sst sich mit der NoSQL-Strategie eine flexible Datenstruktur erstellen. Das macht im Falle von \arbeitstitel \ besonders Sinn, da sich die Art und der Umfang der Daten, die verarbeitet werden, sehr schnell �ndern k�nnen. Au�erdem vereinfacht dies auch die Erweiterbarkeit der Web-Applikation.

\subsection{Frontend mit Jade und Stylus}
Im Frontend werden die Werkzeuge Jade und Stylus verwendet. Jade\footcite{jade} ist eine Node Template Engine und vereinfacht das Schreiben von HTML-Code, indem es eine eigene Syntax implementiert. �ber die vereinfachte Schreibweise hinaus implementiert sie auch Iterationen, Bedingungen und Filter. Au�erdem k�nnen sogenannte Mixins implementiert und somit dessen Funktionalit�t wiederverwendet werden.

Stylus\footcite{stylus} ist eine f�r Node entwickelte Sprache, deren Kompilat CSS ist. Die Syntax ist im Gegensatz zu CSS stark vereinfacht. Weiterhin bietet sie Funktionen wie die Deklaration von Variablen, Funktionen und Mixins.

\subsection{Code-Generierung mit Yeoman}
Das Ger�st der Web-Applikation wurde mittels \textbf{Yeoman},\footcite{yeoman} einem Scaffolding-Tool f�r Webapps, generiert. Der Workflow von Yeoman basiert auf den drei Tools \textbf{Yo}, \textbf{Grunt}\footcite{grunt} und \textbf{Bower}.\footcite{bower} Bower ist ein Package Manager f�r das Frontend, Grunt ein Task Runner, mit Hilfe dessen Automatisierungsprozesse und Build-Skripte ausgef�hrt werden k�nnen. Yo kann mittles verschiedener Generatoren und Konfigurationen den Boilerplate-Code einer Web-Applikation und ein passendes Grunt Skript generieren.

Im Falle von \arbeitstitel \ wurde der Yeoman-Generator \textbf{generator-express}\footcite{yeoExpress} verwendet. Das generierte Projekt bereits aus einer MVC-Struktur und hat bereits MongoDB als Default-Datenbank vorkonfiguriert. 
\begin{itemize}
	\item NodeJS
	\item Yeoman
	\item Grunt
	\item Bower
	\item Stylus
	\item mongoose
\end{itemize}

\section{Tools}
Im Folgenden werden verwendete Libraries und deren Nutzern erl�utert.

\subsection{Select2}

\subsection{async}
Das async-Modul erlaubt eine bessere Kontrolle �ber asynchron ausgef�hrte Funktionalit�t. Da \arbeitstitel \ an vielen Stellen mit mehreren APIs kommunizieren muss, ist die Asynchronit�t zwar generell ein Vorteil, jedoch werden auch Anfragen ben�tigt, die von mehreren asynchronen Requests abh�ngig sind. \Dahe , dass einige Requests von teilweise mehreren Requests, die vorher durchgef�hrt werden m�ssen, abh�ngig sind. Mit async lassen sich Callbacks in vordefinierter Reihenfolge durchf�hren. Auch eine Serie von ein und dem selben Callbacks auf Werten aus einer Liste, ist m�glich, genauso lassen sich mehrere Requests gleichzeitig abarbeiten bis dann der finale Callback ausgef�hrt werden kann, wenn alle Vorbedingungen erf�llt worden sind.  

\subsection{node-rest-client}


\subsection{cheerio}
\subsection{nodebrainz und coverart}

\subsection{node-imdb-api}


\section{Grobkonzept}

\section{Feinkonzept}

\section{Arbeitsweise}
\medskip
\begin{lstlisting}[caption=My Javascript Example]
Name.prototype = {
	methodName: function(params){
		var doubleQuoteString = "some text";
		var singleQuoteString = 'some more text';
		// this is a comment
		if(this.confirmed != null && typeof(this.confirmed) == Boolean && this.confirmed == true){
			document.createElement('h3');
			$('#system').append("This looks great");
			return false;
		} else {
			throw new Error;
		}
	}
}
\end{lstlisting}