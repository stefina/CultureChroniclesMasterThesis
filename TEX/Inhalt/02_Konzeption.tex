%!TEX root = ../Masterarbeit.tex
\chapter{Konzeption}
\label{cha:konzeption}
Im Folgenden wird das Konzept der Web-Applikation erarbeitet und vorgestellt. Daf�r wird vorerst ein Fernkonzept und ein dazugeh�riges Mockup erstellt, wobei das Fernkonzept �ber den gesamten w�nschenswerten Umfang einer solchen Applikation verf�gen soll.  Das bedeutet, dass die Applikation nicht auf m�glichen Medieninhalte und -typen beschr�nkt werden soll, sondern den heute und in naher Zukunft m�glichen Spielraum ausnutzen soll.
Auf dieser Basis wird ein Mockup entwickelt, welches die Anwendungsf�lle und Darstellungsm�glichkeiten beinhaltet.
Anschlie�end wird das Fernkonzept auf ein im Rahmen dieser Masterarbeit realisierbaren Umfang reduziert und entsprechend eingegrenzt. Am zuvor erstellten Mockup sollten diesbez�glich nur wenige �nderungen \bzw Einschr�nkungen vorgenommen werden m�ssen.

\section{Fernkonzept}

\section{Mockup}


\subsection{Funktionalit�t}
Fokus der Applikation ist eine komfortable Suche, die dem Nutzer m�glichst einfach zug�nglich gemacht wird, \dahe vom Design, als auch von der Funktionalit�t. 
Eine f�r diese Web-Applikation besondere Aufgabe ist es, die drei Kernfunktionen m�glichst gleichzeitig gut erreichbar zu machen, ohne die Gesamterscheinung zu st�ren.
\paragraph{Suchen}

\paragraph{Browsen}

\paragraph{Konsumieren}
Unter den dargestellten Inhalten befinden sich auch abspielbare Inhalte. Diese soll der Nutzer sich m�glichst gleichzeitig anschauen k�nnen


Fokus der Applikation ist eine komfortable Suche, anhand dessen der Nutzer keine falschen Vorstellungen �ber die gelieferten Ergebnisse bekommt. \todo{doofer Satz, aber vielleicht nur die falsche Stelle}

\subsection{Design und Layout}
Da die Web-Applikation sehr viele unterschiedliche Medienarten und -inhalte darstellen soll, ist ein sauberes aufger�umtes Design umso wichtiger, um es dem Nutzer zu erm�glichen, sich auf die Inhalte zu konzentrieren. Darum ist es wichtig, dass das Design m�glichst wandelbar ist, unabh�ngig vom Inhalt. Folgende Aspekte sind beim Layout zu beachten:

\paragraph{Suche und Filter}

\paragraph{Musik-Player}

\paragraph{Video-Player}


\section{Eingrenzung}