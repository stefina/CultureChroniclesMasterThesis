%!TEX root = ../Masterarbeit.tex
\chapter{Ergebnis}
\label{cha:ergebnis}
Im Folgenden wird der Funktionsumfang und die Anwendung der Web-Applikation beschrieben.

\section{Anwendung}


\begin{figure}[htb]
	\begin{center}
		\shadowimage[width=14cm]{screenHome.png}
		\caption{Screenshot des Homescreens}
	\end{center}
	\label{fig:screenHome}
\end{figure}


\begin{figure}[htb]
	\begin{center}
		\shadowimage[width=14cm]{screenSuggestionsGood.png}
		\caption{Screenshot der Anzeige der Suchvorschl�ge}
	\end{center}
	\label{fig:screenSuggestions}
\end{figure}


\begin{figure}[htb]
	\begin{center}
		\shadowimage[width=14cm]{screenSearchResults1990.png}
		\caption{Screenshot der Anzeige der Suchergebnisse des Jahres 1990}
	\end{center}
	\label{fig:screenSearchResults}
\end{figure}


\subsection{Eingabe}
jahr (nicht datum)


\section{Funktionalit�t}


\begin{itemize}
    \item screenshots vom Prototypen
    \item was ist gut gelaufen, was ist schlecht gelaufen, wo wurde gespart
\end{itemize}