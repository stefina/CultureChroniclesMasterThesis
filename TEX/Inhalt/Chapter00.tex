%!TEX root = ../Masterarbeit.tex
\chapter{Einleitung}
\label{cha:Einleitung}

\section{Motivation}
Die Idee f�r diese Arbeit ist beim H�ren eines �lteren Musiktitels entstanden. Hierbei entstand der Wunsch danach zu diesem Zeitpunkt einen Radiosender h�ren zu k�nnen, wie er sich zu der Zeit der Popularit�t des Musiktitels angeh�rt h�tte. Dabei m�sste in Betracht gezogen werden, welche Lieder bereits zu diesem Zeitpunkt popul�r gewesen sind, bzw. welche zuk�nftigen Hits vielleicht schon im Radio gespielt worden sind. Dies allein l�sst sich mehr oder weniger simpel mit alten Musiksamplern abdecken. Interessant wird es jedoch dann, wenn �ber die Musik hinaus auch zus�tzliche Arten von Medien mit in Betracht gezogen werden.

\section{Ziel und Abgrenzung dieser Arbeit}
Die Herausforderung dieser Arbeit ist es, eine Web-Anwendung zu implementieren, die es dem Nutzer erlaubt anhand eines Datums, Informationen verschiedener Medientypen aus dieser Zeit anzeigen zu lassen. Diese Informationen sollen sowohl Musik, als auch Bilder, Videos und Neuigkeiten beinhalten. Hierbei sollen zeitlich voneinander abh�ngige und thematisch verwandte Daten dargestellt werden. So kann \zB ein zu der Zeit aufgetretenes wichtiges Ereignis in Form eines Wikipedia-Artikels dargestellt werden, w�hrend die passende Musik, sowie Verweise popul�re Filme und ggf. Bilder, die den Stil der Zeit darstellen.
Dabei besteht das Ziel, die Web-Applikation m�glichst einfach erweiterbar zu machen, um bspw. neue Medientypen und Quellen einzubinden und das Look-and-feel variabel zu halten.

\section{Aufbau und Vorgehensweise}
Der Entwicklung der Web-Applikation wird zun�chst eine Feldanalyse vorangestellt, die potenziell �hnliche Projekte und Ans�tze behandeln soll. Daraufhin soll eine Feature-Liste erstellt werden, die m�gliche Use Cases des Nutzers vorstellt und Beispiel-Ergebnisse zeigt, um die Funktionsweise und den Zweck der Applikation besser abgrenzen zu k�nnen. 
Es folgt eine Er�rterung der Medien-Typen und Arten von Daten, die in solch einer Applikation von Nutzen oder Interesse sein k�nnten. Im weiteren Verlauf werden m�gliche APIs und andere Quellen recherchiert und nach der Verf�gbarkeit der gew�nschten Daten und nach dem Nutzen f�r diese Arbeit bewertet um schlie�lich eine Auswahl zu treffen, welche Daten in die Applikation eingebunden werden sollen.
Im Anschluss darauf wird ein Grobkonzept erstellt woraufhin passende Hilfsmittel (\zB Frameworks) zur Implementierung der Web-Applikation ausgew�hlt werden. Daraufhin wird das Grobkonzept verfeinert.
Zum Schluss wird die Herangehensweise und das Ergebnis dieser Arbeit reflektiert und ein Fazit gezogen.
