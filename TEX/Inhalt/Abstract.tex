%!TEX root = ../Masterarbeit.tex
\section*{Zusammenfassung}
\label{sec:Zusammenfassung}
Diese Arbeit befasst sich mit der Aggregation von Kulturg�tern anhand eines w�hlbaren Zeitpunktes. Es wird analysiert inwiefern Media-Content verschiedener Medientypen von unterschiedlichen Quellen des Web aggregiert werden kann, um einen Querschnitts des gew�hlten Zeitpunktes anzuzeigen. Hauptteil dieser Arbeit ist die Konzeption einer Web-Applikation, die Daten in Form von Bild, Ton, Video und Text aus einem gew�hlten Zeitraum erfassen und pr�sentieren kann. Dabei kann der Zeitraum \ua mithilfe eines Musikst�ckes, Filmes, Ereignisses \oae Suchbegriff, der sich zu einem Datum aufl�sen l�sst, definiert werden. Im Rahmen dieser Arbeit wird ein Prototyp in Form einer funktionsf�higen Web-Applikation als Machbarkeitsnachweis implementiert. An dieser Arbeit soll der M�glichkeitsspielraum des Internets bez�glich einer zeitspezifischen Recherche analysiert werden.

\section*{Abstract}
\label{sec:Abstract}
This work deals with the retrieval of cultural goods on the basis of an arbitrary point in time. It will analyze how media content of various types of media from different sources of the Web can be aggregated to show a cross section of the selected time point. Main part of this work is to design a web application that can capture the data in the form of image, sound, video and text from a selected time period. In this case, the period can be specified using a piece of music, movie, event or a similar searchterm that can be resolved to a date. In this work, a prototype in the form of a functional web application is implemented as a proof of concept. In this work, the possibilities of a time-specific research will be analyzed.
