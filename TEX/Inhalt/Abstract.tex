%!TEX root = ../Masterarbeit.tex
\section*{Zusammenfassung}
\label{sec:Zusammenfassung}
Diese Arbeit befasst sich mit der Aggregation von Kulturg�tern anhand eines w�hlbaren Zeitpunktes. Es wird analysiert inwiefern Media-Content verschiedener Medientypen von unterschiedlichen Quellen des Web aggregiert werden kann, um einen Querschnitts des gew�hlten Zeitpunktes anzuzeigen. Hauptteil dieser Arbeit ist die Konzeption einer Web-Applikation, die Daten in Form von Bild, Ton, Video und Text aus einem gew�hlten Zeitraum erfassen und pr�sentieren kann. Dabei kann der Zeitraum \ua mithilfe eines Musikst�ckes, Filmes, Ereignisses \oae Suchbegriff, der sich zu einem Datum aufl�sen l�sst, definiert werden. Im Rahmen dieser Arbeit wird ein Prototyp in Form einer funktionsf�higen Web-Applikation als Machbarkeitsnachweis implementiert. An dieser Arbeit soll der M�glichkeitsspielraum des Internets bez�glich einer zeitspezifischen Recherche analysiert werden.

\section*{Abstract}
\label{sec:Abstract}
The main purpose of this thesis is to design a web application that can display data in the form of image, sound, video and text from a selected time period. This thesis will discuss the possibilities of using time as a search filter to produce relevant results. Furthermore, it will consider how information like music, movies or art, can be retrieved during a given time frame. In this case, the time period can be specified using a song, movie, event or a similar search term that can be resolved to a date. It will also analyse how various forms of media from the web can be collated and cross examined to produce a search result. Finally, evidence of this concept will be presented by implementing a functional web application.


