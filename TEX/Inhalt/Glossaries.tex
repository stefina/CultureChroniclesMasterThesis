%!TEX root = ../Masterarbeit.tex
%\newglossaryentry{electrolyte}{name=electrolyte, description={solution able to conduct electric current}}
\newglossaryentry{ID3}{name=ID3-Tag, description={Datenformat, das in MP3-Dateien vorhanden sein kann und Meta-Informationen, wie \zB Interpret und Titel eines Musikst�ckes, enth�lt. Dar�ber hinaus k�nnen auch Informationen wie Musik-Genre oder Songtexte im ID3 gespeichert sein}}
\newglossaryentry{Metadaten}{name=Metadaten, description={Metadaten enthalten Informationen �ber einen Datensatz, ohne �ber die Daten selbst zu verf�gen}}
\newglossaryentry{Datentyp}{name=Datentyp, description={Nach allgemeiner Definition gibt der Datentyp Aufschluss dar�ber, wie ein Datensatz zu verarbeiten ist. Dies ist beispielsweise an der Dateiendung ersichtlich}}
\newglossaryentry{Datenform}{name=Datenform, description={In dieser Arbeit meint der Begriff Datenform die Art, in der ein Datentyp vorliegt \bzw wie dieser gepflegt ist. So k�nnen zwei Dateien vom gleichen Datentyp in unterschiedlichen Formen vorliegen, wenn \bspw unterschiedliche Werte wie Titel und Interpret vorhanden sind, jedoch Informationen �ber Erscheinungsdatum fehlen. Ebenso sind unterschiedliche Varianten der Datenstruktur, w�hrend der Inhalt dennoch gleich sein kann, voneinander unterschiedlich}}
\newglossaryentry{Medientyp}{name=Medientyp, description={In diesem Kontext unterscheidet der Medientyp zwischen Audio, Video, Text, u.s.w}}
\newglossaryentry{Release}{name=Release, description={Ver�ffentlichung. API-bezogen jedes Medium, das als Einheit erworben werden kann. Z.B. ein Album oder eine Single auf CD, wobei jede einzelne Version ein eigenes Release darstellt}}
\newglossaryentry{ReleaseGroup}{name=Release-Group, description={Eine Sammlung von Musik-Ver�ffentlichungen, die bspw. ein Album oder eine Single, die in verschiedenen Staaten erschienen sind, unter einem Begriff zusammen fassen. Ein Album, welches bspw. auf CD und Vinyl jeweils in drei L�ndern erscheint, besteht aus sechs Releases}}
\newglossaryentry{RESTful}{name=RESTful, description={\textbf{Re}presentational \textbf{S}tate \textbf{T}ransfer. RESTful bezeichnet einen Standard von Web-Anwendungen, der f�r die Entwicklung und das Verwenden von Schnittstellen von Wichtigkeit ist}}
\newglossaryentry{DOM}{name=DOM, description={\textbf{D}ocument \textbf{O}bject \textbf{M}odel}}
\newglossaryentry{Medium}{name=Medium, description={Soweit nicht anders angegeben, ist der Begriff Medium und der Begriff der Medien im Sinne der Quarti�ren Medien nach \citep{Fassler1997} zu verstehen. Das bedeutet, dass ein Medium \zB eine Schallplatte, einen Film \oae bezeichnet}}
\newglossaryentry{Tag}{name=Tag, description={Ein Tag ist ein Begriff, der einen Datenbestand auszeichnet. In dieser Arbeit sind haupts�chlich Tags gemeint, die Musikst�cke, Filme u.�.\ bezeichnen und deren Eigenschaften oder Themen-Zugeh�rigkeiten ausdr�cken}}
\newglossaryentry{WYSIWYG}{name=WYSIWYG, description={Akronym f�r \glqq \textbf{W}hat \textbf{Y}ou \textbf{S}ee \textbf{I}s \textbf{W}hat \textbf{Y}ou \textbf{G}et\grqq \ welches im Zusammenhang von Editoren, die ein bearbeitetes Dokument in Echtzeit anzeigen, wie es letztendlich dargestellt werden wird. \citep[Vgl.][]{wiki:wysiwig}}}
\newglossaryentry{npm}{name=npm, description={Package Manager f�r Node.js}}
\newglossaryentry{PageRank}{name=PageRank, description={Der PageRank ist ein Algorithmus, der von der Google Suche verwendet wird, um die Relevanz einer Webseite zu messen. \citep[Vgl.][]{wikiPageRank}}} 