%!TEX root = ../Masterarbeit.tex
%\newglossaryentry{electrolyte}{name=electrolyte, description={solution able to conduct electric current}}
\newglossaryentry{ID3}{name=ID3-Tag, description={Datenformat, das in MP3-Dateien vorhanden sein kann und Meta-Informationen enthalten kann}}
\newglossaryentry{Metadaten}{name=Metadaten, description={Metadaten enthalten Informationen �ber einen Datensatz, ohne �ber die Daten selbst zu verf�gen}}
\newglossaryentry{Datentyp}{name=Datentyp, description={}}
\newglossaryentry{Datenform}{name=Datenform, description={}}
\newglossaryentry{Medientyp}{name=Medientyp, description={In diesem Kontext unterscheidet der Medientyp zwischen Audio, Video, Text, u.s.w}}
\newglossaryentry{Release}{name=Release, description={Ver�ffentlichung. API-bezogen jedes Medium, das als Einheit erworben werden kann. Z.B. ein Album oder eine Single auf CD, wobei jede einzelne Version ein eigenes Release darstellt}}
\newglossaryentry{ReleaseGroup}{name=Release-Group, description={Eine Sammlung von Musik-Ver�ffentlichungen, die bspw. ein Album oder eine Single, die in verschiedenen Staaten erschienen sind, unter einem Begriff zusammen fassen. Ein Album, welches bspw. auf CD und Vinyl jeweils in drei L�ndern erscheint, besteht aus sechs Releases}}
\newglossaryentry{RESTful}{name=RESTful, description={Representational State Transfer}}
\newglossaryentry{DOM}{name=DOM, description={Document Object Model}}
\newglossaryentry{Medium}{name=Medium, description={Soweit nicht anders angegeben, ist der Begriff Medium und der Begriff der Medien im Sinne der Quari�ren Medien nach Fa�ler (1997) zu verstehen. Das bedeutet ...}}
\newglossaryentry{Tag}{name=Tag, description={}}
