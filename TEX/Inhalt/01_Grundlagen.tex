%!TEX root = ../Masterarbeit.tex
\chapter{Grundlagen}
\label{cha:grundlagen}
Zun�chst soll die Ausgangssituation beschrieben werden. (Auf dieser Grundlage werden f�r diese Arbeit wichtige grundlegende Begriffe gekl�rt.) Darauf aufbauend kann der Anforderungsumfang gekl�rt werden.
\todo{Einleitung ggf. l�schen oder umschreiben}

\section{Vorfeldrecherche}
Im Sinne des zeitspezifischen St�berns gibt es bereits verschiedenste Ans�tze.


\subsection{Feldanalyse im Bereich Musik}
Zur dynamischen Zusammenstellung von Musik, insbesondere einer gew�nschten Zeit, gibt es derzeit eine Vielzahl von M�glichkeiten. Sieht man jedoch von Ergebnissen von Musik-Samplern , CD-Sammlungen und selbst erstellten Playlisten ab, da diese meist h�ndisch erstellt worden sind \footcite{googleMusicTimeline}


M�chte der Nutzer eine dynamische Zusammenstellung von Musik h�ren, insbesondere Musik einer von ihm gew�nschten Zeit, so gibt es derzeit eine Vielzahl von M�glichkeiten, die ein mehr oder weniger passendes Musikerlebnis liefern. Der Wunsch nach dynamischen Inhalt bedeutet in diesem Fall, dass der Nutzer nicht auf \bspw Sampler oder eigene CD-Sammlungen oder selbst erstellte Playlisten zugreifen m�chte oder kann.
Im Internet gibt es unz�hlbar viele Dienste, die es dem Nutzer erm�glichen unterschiedlichste Zusammenstellungen von Musiktiteln zu h�ren. Dazu geh�ren \zB �ffentlich verf�gbare Playlisten und Streaming-Dienste. Diese und weitere derzeit verf�gbare M�glichkeiten Musik einer gew�nschten Zeit zu h�ren, werden im Folgenden kurz vorgestellt.

\paragraph{Themenorientierte Radiosendung/Spezielle Radiosender}
Eine der g�ngigsten Methoden ist es, einen Radiosender, der vorwiegend Musik des gew�nschten Zeitraums spielt. So gibt es zahlreiche Radiosender und -shows, die sich auf Musik aus den 80er oder 90er Jahren spezialisiert haben. Dies erm�glicht zwar eine meist zuverl�ssige Eingrenzung f�r einen bestimmten Zeitraum, jedoch gibt es in dieser Form kein Angebot, das sich auf einen genaueren Zeitraum beschr�nkt. Zwar gibt es \ua spezielle Shows, die Musik eines genaueren Zeitraums spielen (\bspw eine Radioshow, die sich konkret mit einem bestimmten Jahr befasst), doch sind diese meist zeitlich begrenzt und nicht auf Anfrage abrufbar.

\paragraph{Online-Streaming anhand eines Tags}
Bei zahlreichen Online-Streaming-Diensten kann der Nutzer Musiksender anhand eines Tags erstellen lassen. Dies k�nnen \zB Tags wie \glqq 1970\grqq, \glqq 70er\grqq, \glqq seventies\grqq, \glqq siebziger\grqq  oder \glqq 70\grqq sein. Diese Tags werden jedoch meist von Usern dieser Plattformen gepflegt, \dahe dass diese unvollst�ndig oder fehlerhaft sein k�nnen. Au�erdem ist bereits an der Vielzahl der verschiedenen Tags mit der gleichen Bedeutung sichtbar, wie zuverl�ssig solche Tags sein k�nnen. Sich dar�ber hinaus Musik eines bestimmten Jahres oder sogar eines genaueren Zeitpunktes anh�ren zu wollen, ist dementsprechend noch unzuverl�ssiger.


\subsection{Feldanalyse im Bereich News}

\subsection{Feldanalyse im Bereich Bild/Kunst}


\section{Allgemeine Probleme bei zeitabh�ngigen Medien}
\paragraph{Schwelle zwischen zwei Jahrzehnten}
Sich einen 80er-Radiosender anzuh�ren, ist nicht zwingend die beste L�sung, sollte \zB Musik aus dem Jahr 1991 gew�nscht sein, sind Musiktitel aus dem Jahre 1989 aufgrund der zeitlichen N�he meist relevanter als Musiktitel aus dem Jahre 1999.

\section{Analyse}
\subsection{Anforderungsanalyse}
\paragraph{Suche und Filtern}


\subsection{Feature-Liste}
\paragraph{Musik}
\paragraph{Video}
\paragraph{Bild}
\paragraph{News/Politik}
\paragraph{Demografische Daten}


