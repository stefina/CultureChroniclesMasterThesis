%!TEX root = Masterarbeit.tex
% Anpassung des Seitenlayouts --------------------------------------------------
%   siehe Seitenstil.tex
% ------------------------------------------------------------------------------
\usepackage[
    automark, % Kapitelangaben in Kopfzeile automatisch erstellen
    headsepline, % Trennlinie unter Kopfzeile
    ilines % Trennlinie linksb�ndig ausrichten
]{scrpage2}

% Anpassung an Landessprache ---------------------------------------------------
\usepackage[ngerman]{babel}

% Umlaute ----------------------------------------------------------------------
%   Umlaute/Sonderzeichen wie ���� direkt im Quelltext verwenden (CodePage).
%   Erlaubt automatische Trennung von Worten mit Umlauten.
% ------------------------------------------------------------------------------
\usepackage[latin1]{inputenc}
\usepackage[T1]{fontenc}
\usepackage{textcomp} % Euro-Zeichen etc.

% Schrift ----------------------------------------------------------------------
\usepackage{lmodern} % bessere Fonts
\usepackage{relsize} % Schriftgr��e relativ festlegen

% Grafiken ---------------------------------------------------------------------
% Einbinden von JPG-Grafiken erm�glichen
\usepackage[dvips,final]{graphicx}
% hier liegen die Bilder des Dokuments
\graphicspath{{Bilder/}}

% Befehle aus AMSTeX f�r mathematische Symbole z.B. \boldsymbol \mathbb --------
\usepackage{amsmath,amsfonts}

% f�r Index-Ausgabe mit \printindex --------------------------------------------
\usepackage{makeidx}

% Einfache Definition der Zeilenabst�nde und Seitenr�nder etc. -----------------
\usepackage{setspace}
\usepackage{geometry}

% Symbolverzeichnis ------------------------------------------------------------
%   Symbolverzeichnisse bequem erstellen. Beruht auf MakeIndex:
%     makeindex.exe %Name%.nlo -s nomencl.ist -o %Name%.nls
%   erzeugt dann das Verzeichnis. Dieser Befehl kann z.B. im TeXnicCenter
%   als Postprozessor eingetragen werden, damit er nicht st�ndig manuell
%   ausgef�hrt werden muss.
%   Die Definitionen sind ausgegliedert in die Datei "Glossar.tex".
% ------------------------------------------------------------------------------
\usepackage[intoc]{nomencl}
\let\abbrev\nomenclature
\renewcommand{\nomname}{Abk�rzungsverzeichnis}
\setlength{\nomlabelwidth}{.25\hsize}
\renewcommand{\nomlabel}[1]{#1 \dotfill}
\setlength{\nomitemsep}{-\parsep}

% zum Umflie�en von Bildern ----------------------------------------------------
\usepackage{floatflt}
\usepackage{float}
\restylefloat{table} % soll Tabellen besser platzieren.. seh aber keinen Unterschied -.-


% zum Einbinden von Programmcode -----------------------------------------------
\usepackage{scrhack}
\usepackage{listings}
\lstset{escapeinside={(*@}{@*)}}
\usepackage{xcolor} 
\definecolor{hellgelb}{rgb}{1,1,0.9}
\definecolor{colKeys}{rgb}{0,0,1}
\definecolor{colIdentifier}{rgb}{0,0,0}
\definecolor{colComments}{rgb}{1,0,0}
\definecolor{colString}{rgb}{0,0.5,0}
\lstset{
    float=hbp,
    basicstyle=\ttfamily\color{black}\small\smaller,
    identifierstyle=\color{colIdentifier},
    keywordstyle=\color{colKeys},
    stringstyle=\color{colString},
    commentstyle=\color{colComments},
    columns=flexible,
    tabsize=2,
    frame=single,
    extendedchars=true,
    showspaces=false,
    showstringspaces=false,
    numbers=left,
    numberstyle=\tiny,
    breaklines=true,
    backgroundcolor=\color{hellgelb},
    breakautoindent=true
}

\newcommand{\colorrow}{\rowcolor{lightgreen}}
\newcommand{\colorCell}{\cellcolor{lightgray}}

% URL verlinken, lange URLs umbrechen etc. -------------------------------------
\usepackage{url}

% wichtig f�r korrekte Zitierweise ---------------------------------------------
%\usepackage[square]{natbib}
\usepackage{jurabib}

\jurabibsetup{
	commabeforerest,
%	ibidem=strict,
	authorformat=abbrv,
	citefull=first,
	see,
	titleformat={colonsep,all},
}

\renewcommand*{\jbauthorfont}{\textsc}
\renewcommand*{\biblnfont}{\scshape\textbf}
\renewcommand*{\bibfnfont}{\normalfont\textbf}

% PDF-Optionen -----------------------------------------------------------------
\usepackage[
    bookmarks,
    bookmarksopen=true,
    colorlinks=true,
% diese Farbdefinitionen zeichnen Links im PDF farblich aus
    linkcolor=red, % einfache interne Verkn�pfungen
    anchorcolor=black,% Ankertext
    citecolor=blue, % Verweise auf Literaturverzeichniseintr�ge im Text
    filecolor=magenta, % Verkn�pfungen, die lokale Dateien �ffnen
    menucolor=red, % Acrobat-Men�punkte
    urlcolor=cyan, 
% diese Farbdefinitionen sollten f�r den Druck verwendet werden (alles schwarz)
    %linkcolor=black, % einfache interne Verkn�pfungen
    %anchorcolor=black, % Ankertext
    %citecolor=black, % Verweise auf Literaturverzeichniseintr�ge im Text
    %filecolor=black, % Verkn�pfungen, die lokale Dateien �ffnen
    %menucolor=black, % Acrobat-Men�punkte
    %urlcolor=black, 
    backref,
    plainpages=false, % zur korrekten Erstellung der Bookmarks
    pdfpagelabels, % zur korrekten Erstellung der Bookmarks
    hypertexnames=false, % zur korrekten Erstellung der Bookmarks
    linktocpage % Seitenzahlen anstatt Text im Inhaltsverzeichnis verlinken
]{hyperref}
% Befehle, die Umlaute ausgeben, f�hren zu Fehlern, wenn sie hyperref als Optionen �bergeben werden
\hypersetup{
    pdftitle={\titel \untertitel},
    pdfauthor={\autor},
    pdfcreator={\autor},
    pdfsubject={\titel \untertitel},
    pdfkeywords={\titel \untertitel},
}

\usepackage[nonumberlist]{glossaries}

% fortlaufendes Durchnummerieren der Fu�noten ----------------------------------
\usepackage{chngcntr}

% f�r lange Tabellen -----------------------------------------------------------
\usepackage{longtable}
\usepackage{array}
\usepackage{ragged2e}
\usepackage{lscape}

\usepackage{rotating} 

% Spaltendefinition rechtsb�ndig mit definierter Breite ------------------------
\newcolumntype{w}[1]{>{\raggedleft\hspace{0pt}}p{#1}}
\newcolumntype{R}[1]{>{\raggedleft\hspace{0pt}\arraybackslash}p{#1}}
%\newcolumntype{s}{>{\raggedleft\arraybackslash}p{2,5cm}}

\newcolumntype{P}[1]{>{\hspace{0pt}}p{#1}}

% Formatierung von Listen �ndern -----------------------------------------------
\usepackage{paralist}

% bei der Definition eigener Befehle ben�tigt
\usepackage{ifthen}

% definiert u.a. die Befehle \todo und \listoftodos
\usepackage{todonotes}

% sorgt daf�r, dass Leerzeichen hinter parameterlosen Makros nicht als Makroendezeichen interpretiert werden
\usepackage{xspace}

\usepackage{wrapfig}

% Taken from Lena Herrmann at 
% http://lenaherrmann.net/2010/05/20/javascript-syntax-highlighting-in-the-latex-listings-package
\usepackage{color, colortbl}
\definecolor{lightgray}{rgb}{.9,.9,.9}
%TABLECOLORS
\definecolor{lightyellow}{rgb}{.9,.9,.6} % Music-Table
\definecolor{lightgreen}{rgb}{.6,.8,.5} % Movies-Table
\definecolor{lightblue}{rgb}{.7,.8,.9} % Events-Table

%CODE COLORS
\definecolor{neutralyellow}{rgb}{.9,.7,.2}
\definecolor{neutralgreen}{rgb}{.2,.6,.4}
\definecolor{neutralblue}{rgb}{.1,.2,.6}

% HTW CODE COLORS
\definecolor{code_numbers}{cmyk}{0.55, 0, 1, 0.25} % abgedunkeltes htw gr�n
\definecolor{code_string_comments}{cmyk}{0, 0, 0, 0.7} % dunkelgrau
\definecolor{code_methods}{cmyk}{0, 0.7, 1, 0}% htw orange
\definecolor{code_rosa}{rgb}{0.73,0.38,0.84}




\definecolor{darkgray}{rgb}{.4,.4,.4}
\definecolor{purple}{rgb}{0.65, 0.12, 0.82}
\definecolor{darkgreen}{rgb}{0.44,0.80,0.08}

\lstdefinelanguage{JavaScript}{
  keywords={typeof, new, true, false, catch, function, return, null, catch, switch, var, if, in, while, do, else, case, break},
  keywordstyle=\color{code_numbers}\bfseries,
  ndkeywords={class, export, boolean, throw, implements, import, this},
  ndkeywordstyle=\color{darkgray}\bfseries,
  identifierstyle=\color{black},
  sensitive=false,
  comment=[l]{//},
  morecomment=[s]{/*}{*/},
  commentstyle=\color{gray}\ttfamily,
  stringstyle=\color{code_methods}\ttfamily,
  morestring=[b]',
  morestring=[b]"
}

\lstset{
   language=JavaScript,
   backgroundcolor=\color{white},
   extendedchars=true,
   basicstyle=\footnotesize\ttfamily,
   showstringspaces=false,
   showspaces=false,
   numbers=left,
   numberstyle=\footnotesize,
   numbersep=9pt,
   tabsize=3,
   breaklines=true,
   showtabs=false,
   captionpos=b,
   frame=bt
}

% \usepackage{pdfpages}
% \usepackage{csvsimple}

% Taken from http://tex.stackexchange.com/questions/22240/choosing-font-for-bold-small-caps-or-any-other-particular-familyseriesshape-c
\usepackage{lmodern} \normalfont %to load T1lmr.fd 
\DeclareFontShape{T1}{lmr}{bx}{sc} { <-> ssub * cmr/bx/sc }{}
%\usepackage{charter}




% DROPSHADOW KRAM
\usepackage{tikz}
\usetikzlibrary{shadows,calc}

% code adapted from http://tex.stackexchange.com/a/11483/3954

% some parameters for customization
\def\shadowshift{3pt,-3pt}
\def\shadowradius{6pt}

\colorlet{innercolor}{black!60}
\colorlet{outercolor}{gray!05}

% this draws a shadow under a rectangle node
\newcommand\drawshadow[1]{
    \begin{pgfonlayer}{shadow}
        \shade[outercolor,inner color=innercolor,outer color=outercolor] ($(#1.south west)+(\shadowshift)+(\shadowradius/2,\shadowradius/2)$) circle (\shadowradius);
        \shade[outercolor,inner color=innercolor,outer color=outercolor] ($(#1.north west)+(\shadowshift)+(\shadowradius/2,-\shadowradius/2)$) circle (\shadowradius);
        \shade[outercolor,inner color=innercolor,outer color=outercolor] ($(#1.south east)+(\shadowshift)+(-\shadowradius/2,\shadowradius/2)$) circle (\shadowradius);
        \shade[outercolor,inner color=innercolor,outer color=outercolor] ($(#1.north east)+(\shadowshift)+(-\shadowradius/2,-\shadowradius/2)$) circle (\shadowradius);
        \shade[top color=innercolor,bottom color=outercolor] ($(#1.south west)+(\shadowshift)+(\shadowradius/2,-\shadowradius/2)$) rectangle ($(#1.south east)+(\shadowshift)+(-\shadowradius/2,\shadowradius/2)$);
        \shade[left color=innercolor,right color=outercolor] ($(#1.south east)+(\shadowshift)+(-\shadowradius/2,\shadowradius/2)$) rectangle ($(#1.north east)+(\shadowshift)+(\shadowradius/2,-\shadowradius/2)$);
        \shade[bottom color=innercolor,top color=outercolor] ($(#1.north west)+(\shadowshift)+(\shadowradius/2,-\shadowradius/2)$) rectangle ($(#1.north east)+(\shadowshift)+(-\shadowradius/2,\shadowradius/2)$);
        \shade[outercolor,right color=innercolor,left color=outercolor] ($(#1.south west)+(\shadowshift)+(-\shadowradius/2,\shadowradius/2)$) rectangle ($(#1.north west)+(\shadowshift)+(\shadowradius/2,-\shadowradius/2)$);
        \filldraw ($(#1.south west)+(\shadowshift)+(\shadowradius/2,\shadowradius/2)$) rectangle ($(#1.north east)+(\shadowshift)-(\shadowradius/2,\shadowradius/2)$);
    \end{pgfonlayer}
}

% create a shadow layer, so that we don't need to worry about overdrawing other things
\pgfdeclarelayer{shadow} 
\pgfsetlayers{shadow,main}


\newcommand\shadowimage[2][]{%
\begin{tikzpicture}
\node[anchor=south west,inner sep=0] (image) at (0,0) {\includegraphics[#1]{#2}};
\drawshadow{image}
\end{tikzpicture}}


% Link-Umbruch-Kram
\renewcommand{\UrlBreaks}{\do\/\do\=\do\?\do\.\do\:\do\A\do\B\do\C\do\D\do\E\do\F\do\G\do\H\do\I\do\J\do\K\do\L\do\M\do\N\do\O\do\P\do\Q\do\R\do\S\do\T\do\U\do\V\do\W\do\X\do\Y\do\Z}