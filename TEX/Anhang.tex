%!TEX root = Masterarbeit.tex
\chapter{CD}
\label{app:CD}

\section{Inhalt der beigelegten CD}

\begin{itemize}
	\item Grafiken
		\begin{itemize}
			\item Diagramm\_Allgemein.png
			\item Diagramm\_VietnamKrieg.png
			\item Retrieving.pdf
			\item RetrievingSearchresultsAndPlaylist.pdf
			\item RetrievingSuggestions.pdf
		\end{itemize}
	\item Masterarbeit.pdf
	\item Software
		\begin{itemize}
			\item Node
			\begin{itemize}
				\item node-v0.10.26.pkg
				\item node.exe
			\end{itemize}
			\item MongoDB
			\begin{itemize}
				\item mongodb-osx-x86\_64-2.4.9.tgz
				\item mongodb-win32-x86\_64-2008plus-2.4.9.zip
			\end{itemize}
		\end{itemize}
	\item SourceCode
\end{itemize}

\chapter{Analyse der Suchdienste}
\label{app:matrix}
Auf den folgenden Seiten ist die tabellarische �bersicht der untersuchten Suchdienste in den Bereichen Musik, Film und Ereignissen zu sehen.
\begin{enumerate}
	\item Film
	\begin{itemize}
		\item google.de
		\item en.wikipedia.org
		\item Amazon.com
		\item IMDb.com
		\item IMDb.com - Advanced Search
	\end{itemize}
	\item Musik
	\begin{itemize}
		\item google.de
		\item en.wikipedia.org
		\item Amazon.com
		\item lastfm.de
		\item Spotify App
		\item rateyourmusic.com
	\end{itemize}
	\item Ereignisse
	\begin{itemize}
		\item google.de
		\item en.wikipedia.org
		\item historyorb.com
		\item wolframalpha.com
	\end{itemize}
\end{enumerate}

%!TEX root = ../../Masterarbeit.tex
\begin{landscape}
\begin{longtable}{ll>{\raggedright}p{0.75\textwidth}l>{\raggedright}p{0.75\textwidth}lll}
    Film                  & ~                                                               & ~                                                                                         & ~                                                                                                                                                                                                                          & ~                                           & ~                                                                  \\
    ~                     & ~                                                               & ~                                                                                         & ~                                                                                                                                                                                                                          & ~                                           & ~                                                                  \\
    Zeitraum              & Suchbegriff                                                     & Link zum Suchergebnis                                                                     & �berschrift                                                                                                                                                                                                                & Beschreibung                                & Bewertung                                                          \\
    google.de             & ~                                                               & ~                                                                                         & ~                                                                                                                                                                                                                          & ~                                           & ~                                                                  \\
    Achtziger             & \glqq movies 80s\grqq                                                     & \url{http://www.digitaldreamdoor.com/pages/movie-pages/movie\_80s.html}                         & 100 Greatest Movies of the 1980s                                                                                                                                                                                           & Top 100                                     & \textasteriskcentered \textasteriskcentered      \\
    ~                     & ~                                                               & \url{http://en.wikipedia.org/wiki/1980s\_in\_film}                                              & 1980s in film                                                                                                                                                                                                              & Filmgeschichtlicher Artikel                 & \textasteriskcentered \textasteriskcentered      \\
    ~                     & ~                                                               & \url{http://www.imdb.com/search/title/?release\_date=1980,1989\&title\_type=feature}            & Most Popular Feature Films Released 1980 to 1989                                                                                                                                                                           & Top 50                                      & \textasteriskcentered \textasteriskcentered \textasteriskcentered  \\
    Neunziger             & \glqq movies 90s\grqq                                                    & \url{http://www.digitaldreamdoor.com/pages/movie-pages/movie\_90s.html}                         & 100 Greatest Movies of the 1990s                                                                                                                                                                                           & Top 100                                     & \textasteriskcentered \textasteriskcentered      \\
    ~                     & ~                                                               & \url{http://www.pinterest.com/richhadbroo/100-great-90-s-movies/}                               & 100 great 90's movies                                                                                                                                                                                                      & Pinterest Top 100 mit Filmplakaten          & ~                                                                  \\
    ~                     & ~                                                               & \url{http://www.pinterest.com/roxsan13/best-90s-movies/}                                        & best 90s movies                                                                                                                                                                                                            & Pinterest Liste mit Filmplakaten            & ~                                                                  \\
    Vietnamkrieg          & \glqq movies vietnam war\grqq                                            & \url{http://en.wikipedia.org/wiki/Category:Vietnam\_War\_films}                                 & Category:Vietnam War films                                                                                                                                                                                                 & Alphabetische Liste von Filmen              & \textasteriskcentered \textasteriskcentered      \\
    ~                     & ~                                                               & \url{http://en.wikipedia.org/wiki/Vietnam\_War\_in\_film}                                       & Vietnam War in film                                                                                                                                                                                                        & Liste von Filmen mit dem Thema Vietnamkrieg & \textasteriskcentered \textasteriskcentered      \\
    ~                     & ~                                                               & \url{http://warmovies.about.com/od/TopPicks/tp/Top-10-Vietnam-Films-Of-All-Time.htm}            & Top 10 Vietnam Films of All Time                                                                                                                                                                                           & Subjektive Top Ten Liste                    & \textasteriskcentered                                     \\
    ~                     & ~                                                               & ~                                                                                         & ~                                                                                                                                                                                                                          & ~                                           & ~                                                                  \\
    en.wikipedia.org      & ~                                                               & ~                                                                                         & ~                                                                                                                                                                                                                          & ~                                           & ~                                                                  \\
    Achtziger             & \glqq movie 80s\grqq                                                     & \url{http://en.wikipedia.org/wiki/1980s\_in\_film}                                              & 1980s in film                                                                                                                                                                                                              & Filmgeschichtlicher Artikel                 & \textasteriskcentered \textasteriskcentered      \\
    ~                     & ~                                                               & \url{http://en.wikipedia.org/wiki/Alan\_Hunter\_(VJ)}                                           & Alan Hunter (VJ)                                                                                                                                                                                                           & VJ-Seite                                    & ~                                                                  \\
    ~                     & ~                                                               & \url{http://en.wikipedia.org/wiki/American\_Top\_40}                                            & American Top 40                                                                                                                                                                                                            & Musik-Top 40                                & ~                                                                  \\
    Neunziger             & \glqq movie 90s\grqq                                                     & \url{http://en.wikipedia.org/wiki/TeenNick}                                                     & TeenNick                                                                                                                                                                                                                   & TV Sender                                   & ~                                                                  \\
    ~                     & ~                                                               & \url{http://en.wikipedia.org/wiki/R.\_Kelly}                                                    & R. Kelly                                                                                                                                                                                                                   & Musikerseite                                & ~                                                                  \\
    ~                     & ~                                                               & \url{http://en.wikipedia.org/wiki/Ace\_of\_Base}                                                & Ace Of Base                                                                                                                                                                                                                & Bandseite                                   & ~                                                                  \\
    Vietnamkrieg          & \glqq vietnam war music\grqq                                             & \url{http://en.wikipedia.org/wiki/Vietnam\_War\_in\_film}                                       & Vietnam War in film                                                                                                                                                                                                        & Liste von Filmen mit dem Thema Vietnamkrieg & \textasteriskcentered \textasteriskcentered      \\
    ~                     & ~                                                               & \url{http://en.wikipedia.org/wiki/Vietnam\_War}                                                 & Vietnam War                                                                                                                                                                                                                & Artikel �ber den Vietnamkrieg               & ~                                                                  \\
    ~                     & ~                                                               & \url{http://en.wikipedia.org/wiki/Apocalypse\_Now}                                              & Apocalypse Now                                                                                                                                                                                                             & Filmseite                                   & \textasteriskcentered                                     \\
    ~                     & ~                                                               & ~                                                                                         & ~                                                                                                                                                                                                                          & ~                                           & ~                                                                  \\
    Amazon.com            & ~                                                               & ~                                                                                         & ~                                                                                                                                                                                                                          & ~                                           & ~                                                                  \\
    Achtziger             & \glqq 80s\grqq  mit Filter \glqq Movies \& TV\grqq                                 & \url{http://www.amazon.com/Spotlight-Collection-Breakfast-Universals-Anniversary/dp/B006TTC50Y} & 80s Comedies Spotlight Collection [The Breakfast Club, Sixteen Candles, Fast Times at Ridgemont High] (Universal's 100th Anniversary) (1982)                                                                               & Filmsammlung                                & \textasteriskcentered \textasteriskcentered      \\
    ~                     & ~                                                               & \url{http://www.amazon.com/Pure-80s-Ultimate-DVD-Region/dp/B000FUTUY2}                          & Pure '80s: The Ultimate DVD Box [Region 1] (2006)                                                                                                                                                                          & Filmsammlung                                & \textasteriskcentered                                     \\
    ~                     & ~                                                               & \url{http://www.amazon.com/Excellent-Eighties-Demi-Moore/dp/B008L0YN0Y}                         & Excellent Eighties (1980)                                                                                                                                                                                                  & Filmsammlung                                & \textasteriskcentered                                     \\
    Neunziger             & \glqq 90s\grqq mit Filter \glqq Movies \& TV\grqq                                 & \url{http://www.amazon.com/WWE-Greatest-Stars-90s-Rock/dp/B001PPLJOU}                           & WWE: Greatest Stars of the '90s (2009)                                                                                                                                                                                     & DVD-Box der gr��ten Stars der 80er          & ~                                                                  \\
    ~                     & ~                                                               & \url{http://www.amazon.com/Costello-Society-Naughty-Nineties-Privates/dp/B0001FGBZM}            & The Best of Abbott \& Costello, Vol. 2 (Hit the Ice / In Society / Here Come the Co-Eds / The Naughty Nineties / Little Giant / The Time of Their Lives / Buck Privates Come Home / The Wistful Widow of Wagon Gap) (1943) & Filmsammlung �ber Abbott und Castello       & \textasteriskcentered                                     \\
    ~                     & ~                                                               & \url{http://www.amazon.com/90s-Night-8-Movie-Will-Ferrell/dp/B00GOC74WA}                        & 90s Night In - 8-Movie Set                                                                                                                                                                                                 & Filmsammlung                                & \textasteriskcentered \textasteriskcentered      \\
    Vietnamkrieg          & \glqq vietnam war\grqq mit Filter \glqq Movies \& TV\grqq                         & \url{http://www.amazon.com/Vietnam-War-Tour-Duty-Patrol/dp/B004P1DWYG}                          & The Vietnam War Tour of Duty on Patrol                                                                                                                                                                                     & Dokumentation                               & \textasteriskcentered                                     \\
    ~                     & ~                                                               & \url{http://www.amazon.com/gp/product/B0064YPMYK}                                               & Vietnam in HD Season 1, Ep. 1 \glqq The Beginning (1964-1965)\grqq                                                                                                                                                                  & Dokumentation                               & \textasteriskcentered                                     \\
    ~                     & ~                                                               & \url{http://www.amazon.com/Vietnam-Ten-Thousand-Day-War/dp/B00ARWXJ54}                          & Vietnam: The Ten Thousand Day War                                                                                                                                                                                          & Film                                        & \textasteriskcentered                                     \\
    ~                     & ~                                                               & ~                                                                                         & ~                                                                                                                                                                                                                          & ~                                           & ~                                                                  \\
    imdb.com              & ~                                                               & ~                                                                                         & ~                                                                                                                                                                                                                          & ~                                           & ~                                                                  \\
    Achtziger             & \glqq 80s\grqq                                                           & \url{http://www.imdb.com/title/tt0480953/}                                                      & 80s (2005- )                                                                                                                                                                                                               & TV-Serie                                    & ~                                                                  \\
    ~                     & ~                                                               & \url{http://www.imdb.com/title/tt2805998}                                                       & Talk Stoop with Cat Greenleaf                                                                                                                                                                                              & TV-Serie                                    & ~                                                                  \\
    ~                     & ~                                                               & \url{http://www.imdb.com/title/tt0305472}                                                       & That \grq 80s Show (2002- )                                                                                                                                                                                                    & TV-Serie                                    & ~                                                                  \\
    Neunziger             & \glqq 90s\grqq                                                           & \url{http://www.imdb.com/title/tt0415430}                                                       & I Love the \grq 90s (2004- )                                                                                                                                                                                                   & TV-Serie                                    & ~                                                                  \\
    ~                     & ~                                                               & \url{http://www.imdb.com/title/tt3223572}                                                       & The Big Fat Quiz of the 90s (2013)                                                                                                                                                                                         & TV-Movie                                    & \textasteriskcentered                                     \\
    ~                     & ~                                                               & \url{http://www.imdb.com/title/tt1003289}                                                       & Saturday Night Live in the '90s: Pop Culture Nation (2007)                                                                                                                                                                 & TV-Movie                                    & \textasteriskcentered                                     \\
    Vietnamkrieg          & \glqq vietnam war\grqq                                                   & \url{http://www.imdb.com/title/tt0793563}                                                       & Vietnam War (2002)                                                                                                                                                                                                         & Video Game                                  & ~                                                                  \\
    ~                     & ~                                                               & \url{http://www.imdb.com/title/tt0098597}                                                       & Vietnam War Story: The Last Days                                                                                                                                                                                           & Dokumentation                               & ~                                                                  \\
    ~                     & ~                                                               & \url{http://www.imdb.com/title/tt0092475}                                                       & Vietnam War Story (1987-1988)                                                                                                                                                                                              & TV-Serie                                    & ~                                                                  \\
    ~                     & ~                                                               & ~                                                                                         & ~                                                                                                                                                                                                                          & ~                                           & ~                                                                  \\
    imdb.com/search/title & ~                                                               & ~                                                                                         & ~                                                                                                                                                                                                                          & ~                                           & ~                                                                  \\
    Achtziger             & Release Date: 1980-1989\\Title Type: \\Feature Film, TV Movie\\ & \url{http://www.imdb.com/title/tt0093870}                                                       & RoboCop (1987)                                                                                                                                                                                                             & Film                                        & \textasteriskcentered \textasteriskcentered \textasteriskcentered  \\
    ~                     & ~                                                               & \url{http://www.imdb.com/title/tt0093779}                                                       & Die Braut des Prinzen (1987)                                                                                                                                                                                               & Film                                        & \textasteriskcentered \textasteriskcentered \textasteriskcentered  \\
    ~                     & ~                                                               & \url{http://www.imdb.com/title/tt0081505}                                                       & Shining (1980)                                                                                                                                                                                                             & Film                                        & \textasteriskcentered \textasteriskcentered \textasteriskcentered  \\
    Neunziger             & Release Date: 1990-1999\\Title Type: \\Feature Film, TV Movie\\ & \url{http://www.imdb.com/title/tt0111161}                                                       & Die Verurteilten (1994)                                                                                                                                                                                                    & Film                                        & \textasteriskcentered \textasteriskcentered \textasteriskcentered  \\
    ~                     & ~                                                               & \url{http://www.imdb.com/title/tt0120338}                                                       & Titanic (1997)                                                                                                                                                                                                             & Film                                        & \textasteriskcentered \textasteriskcentered \textasteriskcentered  \\
    ~                     & ~                                                               & \url{http://www.imdb.com/title/tt0137523}                                                       & Fight Club (1999)                                                                                                                                                                                                          & Film                                        & \textasteriskcentered \textasteriskcentered \textasteriskcentered  \\
\end{longtable}
\end{landscape}
% Relevanz
% Qualit�t
% automatische Auswertbarkeit
%!TEX root = ../../Masterarbeit.tex
\begin{landscape}
\section*{Matrix der analysierten Suchdienste im Bereich Musik}\label{app:musik}
\begin{longtable}{|R{2.5cm}||R{2.5cm}|R{6.0cm}|R{4.0cm}|R{2.5cm}|l|}%{llllll}
\hline
    Zeitraum              & Suchbegriff                                                     & Link zum Suchergebnis                                                                     & �berschrift                                                                                                                                                                                                                & Beschreibung                                & Bewertung                                                          \\
\hline \hline
\endhead
    \textbf{google.de}                            & ~                                                         & ~                                                                                                     & ~                                                                                                                               & ~                                                       & ~               \\
    Achtziger                            & \glqq music 80s\grqq                                                & \url{http://www.nme.com/list/100-best-songs-of-the-1980s/266358}                                            & 100 BEST TRACKS OF THE EIGHTIES - NME 60 YEARS OF NEW MUSIC                                                                     & Durch NME zusammengestellte Top 100 Liste               & \textasteriskcentered \textasteriskcentered       \\
    ~                                    & ~                                                         & \url{http://www.80smusicvids.com/}                                                                          & Over 1,000 classic music videos from the 1980\grq s                                                                                 & Liste von Musikvideos aus den Achtzigern                & \textasteriskcentered            \\
    ~                                    & ~                                                         & \url{http://www.lastfm.de/tag/80s}                                                                          & Musik mit dem Tag \glqq 80s\grqq                                                                                                         & Musik mit dem \glqq 80s\grqq -Tag                                 & \textasteriskcentered \textasteriskcentered \textasteriskcentered  \\
    Neunziger                            & \glqq music 90s\grqq                                                & \url{http://top40.about.com/od/hitsofthe90s/tp/top1001990s.-CSM.htm}                                        & This is admittedly a subjective list based on judgements of quality instead of sales figures or radio airplay.                  & Subjektiv kumulierte Top 100 Liste                      & \textasteriskcentered            \\
    ~                                    & ~                                                         & \url{http://www.popculturemadness.com/Entertainment/Decades/90s/Music.html}                                 & PCM\grq s Official List of 1990s Songs By Category                                                                                  & Verschiedene Top Listen nach Kategorien                 & \textasteriskcentered \textasteriskcentered       \\
    ~                                    & ~                                                         & \url{http://www.mtv.co.uk/music/charts/official-uk-countdowns/the-official-top-100-singles-of-the-90s}      & The Official Top 100 Singles of The 90\grq s                                                                                        & Liste von Musikvideos                                   & \textasteriskcentered \textasteriskcentered       \\
    Vietnamkrieg                         & \glqq music vietnam war\grqq                                        & \url{http://en.wikipedia.org/wiki/List\_of\_songs\_about\_the\_Vietnam\_War}                                & List of songs about the Vietnam War                                                                                             & Liste von Liedern �ber den Vietnamkrieg                 & \textasteriskcentered \textasteriskcentered       \\
    ~                                    & ~                                                         & \url{http://en.wikipedia.org/wiki/Miss\_Saigon}                                                             & Miss Saigon                                                                                                                     & Artikel �ber Miss Saigon                                & ~               \\
    ~                                    & ~                                                         & \url{http://www.ichiban1.org/html/music.htm}                                                                & Top 10 Hits of the Vietnam War era                                                                                              & Liste der Top Ten Hits jedes Jahres w�hrend des Krieges & \textasteriskcentered \textasteriskcentered       \\
    David Bowie - Changes                & \glqq david bowie changes music\grqq                                & \url{http://waxing-lyrical.squidoo.com/david-bowie-glam-rock-years}                                         & David Bowie: The Glam Rock Years                                                                                                & musikgeschichtlicher Text                               & ~               \\
    ~                                    & ~                                                         & \url{http://uncyclopedia.wikia.com/wiki/David\_Bowie}                                                       & David Bowie                                                                                                                     & Biographischer Text                                     & ~               \\
    ~                                    & ~                                                         & \url{http://vermontreview.tripod.com/essays/glam.htm}                                                       & Glam Rock: Then and Now                                                                                                         & musikgeschichtlicher Text                               & ~               \\
    \textbf{en.wikipedia.org}                     & ~                                                         & ~                                                                                                     & ~                                                                                                                               & ~                                                       & ~               \\
    Achtziger                            & \glqq music 80s\grqq                                                & \url{http://en.wikipedia.org/wiki/1980s\_in\_music}                                                         & 1980s in music                                                                                                                  & musikgeschichtlicher Text                               & ~               \\
    ~                                    & ~                                                         & \url{http://en.wikipedia.org/wiki/Now\_That\%27s\_What\_I\_Call\_Music!\_discography}                       & Now That\grq s What I Call Music! discography                                                                                       & Liste von Kompilationen                                 & \textasteriskcentered            \\
    ~                                    & ~                                                         & \url{http://en.wikipedia.org/wiki/Now\_That\%27s\_What\_I\_Call\_the\_80s\_(U.S.\_series)}                  & Now That\grq s What I Call the 80s (U.S. series)                                                                                    & Track Liste eines Kompilation Albums der Serie          & \textasteriskcentered            \\
    Neunziger                            & \glqq music 90s\grqq                                                & \url{http://en.wikipedia.org/wiki/1990s\_in\_music}                                                         & 1990s in music                                                                                                                  & musikgeschichtlicher Text                               & ~               \\
    ~                                    & ~                                                         & \url{http://en.wikipedia.org/wiki/Now\_That\%27s\_What\_I\_Call\_Music!\_discography}                       & Now That\grq s What I Call Music! discography                                                                                       & Liste von Kompilationen                                 & \textasteriskcentered            \\
    ~                                    & ~                                                         & \url{http://en.wikipedia.org/wiki/Absolute\_Radio}                                                          & Absolute Radio                                                                                                                  & Artikel �ber einen Radiosender                          & ~               \\
    Vietnamkrieg                         & \glqq music vietnam war\grqq                                        & \url{http://en.wikipedia.org/wiki/United\_States\_Marine\_Corps}                                            & United States Marine Corps                                                                                                      & Artikel �ber die US-Marines                             & ~               \\
    ~                                    & ~                                                         & \url{http://en.wikipedia.org/wiki/Vietnam}                                                                  & Vietnam                                                                                                                         & Artikel �ber Vietnam                                    & ~               \\
    ~                                    & ~                                                         & \url{http://en.wikipedia.org/wiki/Richard\_Nixon}                                                           & Richard Nixon                                                                                                                   & Artikel �ber Richard Nixon                              & ~               \\
    David Bowie - Changes                & \glqq david bowie changes music\grqq                                & \url{http://en.wikipedia.org/wiki/Mick\_Ronson}                                                             & Mick Ronson                                                                                                                     & Artikel �ber Mick Ronson                                & ~               \\
    ~                                    & ~                                                         & \url{http://en.wikipedia.org/wiki/Mick\_Woodmansey}                                                         & Mick Woodmansey                                                                                                                 & Artikel �ber Mick Woodmansey                            & ~               \\
    ~                                    & ~                                                         & \url{http://en.wikipedia.org/wiki/Trident\_Studios}                                                         & Trident Studios                                                                                                                 & Artikel �ber Trident Studios                            & ~               \\
    \textbf{Amazon.de}                            & ~                                                         & ~                                                                                                     & ~                                                                                                                               & ~                                                       & ~               \\
    Achtziger                            & \glqq 80s\grqq  mit Filter \glqq Music\grqq                                   & \url{http://www.amazon.com/3-Pak-80s-Pop-Hits/dp/B00005NKKN}                                                & 3 Pak: 80\grq s Pop Hits [Box Set]                                                                                                  & Kompilation                                             & \textasteriskcentered            \\
    ~                                    & ~                                                         & \url{http://www.amazon.com/Pure-80s-1s/dp/B000ENWKMY}                                                       & Pure 80\grq s \#1s                                                                                                                  & Kompilation                                             & \textasteriskcentered            \\
    ~                                    & ~                                                         & \url{http://www.amazon.com/Big-Hits-80s/dp/B000002T8I}                                                      & Big Hits of 80\grq s                                                                                                                & Kompilation                                             & \textasteriskcentered            \\
    Neunziger                            & \glqq 90s\grqq  mit Filter \glqq Music\grqq                                   & \url{http://www.amazon.com/Forever-90s/dp/B0007YMVC4}                                                       & Forever 90s                                                                                                                     & Kompilation                                             & \textasteriskcentered            \\
    ~                                    & ~                                                         & \url{http://www.amazon.com/100-Essential-Hits-90s/dp/B002R6QUQ8}                                            & 100 Essential Hits of the 90\grq s [Box Set, Import]                                                                                & Kompilation                                             & \textasteriskcentered            \\
    ~                                    & ~                                                         & \url{http://www.amazon.com/Now-Thats-What-Call-1990s/dp/B0043URV46}                                         & Now That\grq s What I Call The 1990s                                                                                              & Kompilation                                             & \textasteriskcentered            \\
    Vietnamkrieg                         & \glqq vietnam war\grqq  mit Filter \glqq Music\grqq                           & \url{http://www.amazon.com/Things-They-Carried-Tim-OBrien-ebook/dp/B002TWIVNA}                              & The Things They Carried [Kindle Edition]                                                                                        & E-Book                                                  & ~               \\
    ~                                    & ~                                                         & \url{http://www.amazon.com/Vietnam-War-Years-1/dp/B0002XEDZ8}                                               & Vietnam War Years 1                                                                                                             & Kompilation                                             & \textasteriskcentered            \\
    ~                                    & ~                                                         & \url{http://www.amazon.com/Country-Folk-Songs-Americans-Vietnam/dp/B000000MQN}                              & In Country: Folk Songs of Americans in the Vietnam War                                                                          & Kompilation                                             & \textasteriskcentered            \\
    David Bowie - Changes                & \glqq david bowie changes\grqq                                      & \url{http://www.amazon.com/Changesbowie-David-Bowie/dp/B00000DTQD}                                          & Changesbowie [Import]                                                                                                           & Album                                                   & ~               \\
    ~                                    & ~                                                         & \url{http://www.amazon.com/Best-David-Bowie/dp/B00006L736}                                                  & Best of David Bowie [Original Recording Remastered]                                                                             & Album                                                   & ~               \\
    ~                                    & ~                                                         & \url{http://www.amazon.com/Hunky-Dory-David-Bowie/dp/B00001OH7O}                                            & Hunky Dory [Enhanced, Original Recording Reissued]                                                                              & Album                                                   & ~               \\
    \textbf{lastfm.de}                            & ~                                                         & ~                                                                                                     & ~                                                                                                                               & ~                                                       & ~               \\
    Achtziger                            & \glqq 80s\grqq                                                      & \url{http://www.lastfm.de/tag/80s}                                                                          & Musik mit dem Tag \glqq 80s\grqq                                                                                                         & Tag-Radio                                               & \textasteriskcentered \textasteriskcentered \textasteriskcentered  \\
    ~                                    & ~                                                         & \url{http://www.lastfm.de/music/Various+Artists/Punk+Goes+80\%27s}                                          & Punk Goes 80\grq s                                                                                                                  & Kompilation                                             & \textasteriskcentered            \\
    ~                                    & ~                                                         & \url{http://www.lastfm.de/music/Miami+Sound+Machine/The+Definitive+80\%27s+(eighties)}                      & The Definitive 80\grq s (eighties)                                                                                                  & Kompilation                                             & \textasteriskcentered            \\
    Neunziger                            & \glqq 90s\grqq                                                      & \url{http://www.lastfm.de/tag/90s}                                                                          & Musik mit dem Tag \glqq 90s\grqq                                                                                                         & Tag-Radio                                               & \textasteriskcentered \textasteriskcentered \textasteriskcentered  \\
    ~                                    & ~                                                         & \url{http://www.lastfm.de/music/Ben+Liebrand/Grandmix:+The+90\%27s+Edition+(Mixed+by+Ben+Liebrand)+(disc+2)} & Grandmix: The 90\grq s Edition (Mixed by Ben Liebrand) (disc 2)                                                                     & Kompilation                                             & ~               \\
    ~                                    & ~                                                         & \url{http://www.lastfm.de/music/Ace+of+Base/Singles+Of+The+90\%27s}                                         & Singles Of The 90\grq s                                                                                                             & Kompilation von Ace Of Base                             & ~               \\
    Vietnamkrieg                         & \glqq vietnam-war\grqq                                              & \url{http://www.lastfm.de/tag/vietnam\%20war}                                                               & Musik mit dem Tag \glqq vietnam war\grqq                                                                                                 & Tag-Radio                                               & \textasteriskcentered            \\
    ~                                    & ~                                                         & \url{http://www.lastfm.de/music/The+Vietnam+War/The+Vietnam+War}                                            & The Vietnam War                                                                                                                 & Album von The Vietnam War                               & ~               \\
    ~                                    & ~                                                         & \url{http://www.lastfm.de/music/Dr.+Sound+Effects/Anti-Vietnam+War+Demonstrations}                          & Anti-Vietnam War Demonstrations                                                                                                 & Album von Dr. Sound Effects                             & ~               \\
    David Bowie - Changes                & \glqq david bowie changes\grqq                                      & \url{http://www.lastfm.de/music/David+Bowie/\_/Changes}                                                     & Changes                                                                                                                         & Song                                                    & ~               \\
    ~                                    & ~                                                         & \url{http://www.lastfm.de/music/David+Bowie/ChangesBowie}                                                   & ChangesBowie                                                                                                                    & Album                                                   & ~               \\
    ~                                    & ~                                                         & \url{http://www.lastfm.de/music/David+Bowie/Changes}                                                        & Changes                                                                                                                         & Song                                                    & ~               \\
    \textbf{Spotify App}                          & ~                                                         & ~                                                                                                     & ~                                                                                                                               & ~                                                       & ~               \\
    Achtziger                            & \glqq 80s\grqq                                                      & \url{http://open.spotify.com/user/johannorin/playlist/58JZmDG1Pe1ppoCEC1aOQp}                               & 80s                                                                                                                             & Playlist                                                & \textasteriskcentered            \\
    ~                                    & ~                                                         & \url{http://open.spotify.com/user/burrjt/playlist/29dTrOurPDrMcrnio2q6hZ}                                   & 80s / Classic Rock                                                                                                              & Playlist                                                & \textasteriskcentered            \\
    ~                                    & ~                                                         & \url{http://open.spotify.com/user/warnerbros.records/playlist/1VgRf3OQ9rhLyuWzyEDfoi}                       & Decades of Rock - 80s                                                                                                           & Playlist                                                & \textasteriskcentered            \\
    Neunziger                            & \glqq 90s\grqq                                                      & \url{http://open.spotify.com/user/1220108378/playlist/7zK2WjuX5otv9au92VXsKc}                               & Ultimate 90s Playlist                                                                                                           & Playlist                                                & \textasteriskcentered            \\
    ~                                    & ~                                                         & \url{http://open.spotify.com/user/sam85uk/playlist/5TcHWbnN6SIhvPY1MXMDrb}                                  & 90s                                                                                                                             & Playlist                                                & \textasteriskcentered            \\
    ~                                    & ~                                                         & \url{http://open.spotify.com/user/myplay.com/playlist/20LKsiDZd4ALrlihncFcFa}                               & 90s Alternative Rock                                                                                                            & Playlist                                                & \textasteriskcentered            \\
    Vietnamkrieg                         & \glqq vietnam war\grqq                                              & \url{http://open.spotify.com/user/1225379340/playlist/54a4Lsi1OdU4T4XV3ysQBU}                               & Vietnam War Era Music                                                                                                           & Playlist                                                & \textasteriskcentered            \\
    ~                                    & ~                                                         & \url{http://open.spotify.com/user/danny265/playlist/6V6ERAyuQvOIooJJLD1WUj}                                 & The best vietnam war songs album eve...                                                                                         & Playlist                                                & \textasteriskcentered            \\
    ~                                    & ~                                                         & \url{http://open.spotify.com/user/mrhumble/playlist/4G4ew8qkJHqFgLGaMzoz71}                                 & Vietnam War - The best playlist in the ...                                                                                      & Playlist                                                & \textasteriskcentered            \\
    David Bowie - Changes                & \glqq david bowie changes\grqq                                      & \url{http://open.spotify.com/user/1238377967/playlist/5b9mWo4PFDqCddejdGLlQB}                               & David Bowie\grq s Changes                                                                                                           & Playlist                                                & \textasteriskcentered            \\
    \textbf{http://rateyourmusic.com/customchart} & ~                                                         & ~                                                                                                     & ~                                                                                                                               & ~                                                       & ~               \\
    Achtziger                            & \glqq 1980s\grqq  mit Filter \glqq Albums\grqq , \glqq As rated by all RYM users \grqq  & \url{http://rateyourmusic.com/release/album/pixies/doolittle/}                                              & Doolittle by Pixies (Album, Alternative Rock): Reviews, Ratings, Credits, Song list - Rate Your Music                           & Album                                                   & \textasteriskcentered \textasteriskcentered \textasteriskcentered  \\
    ~                                    & ~                                                         & \url{http://rateyourmusic.com/release/album/the\_smiths/the\_queen\_is\_dead/}                              & The Queen Is Dead by The Smiths (Album, Jangle Pop): Reviews, Ratings, Credits, Song list - Rate Your Music                     & Album                                                   & \textasteriskcentered \textasteriskcentered \textasteriskcentered  \\
    ~                                    & ~                                                         & \url{http://rateyourmusic.com/release/album/talking\_heads/remain\_in\_light/}                              & Remain in Light by Talking Heads (Album, New Wave): Reviews, Ratings, Credits, Song list - Rate Your Music                      & Album                                                   & \textasteriskcentered \textasteriskcentered \textasteriskcentered  \\
    Neunziger                            & \glqq 1990s\grqq  mit Filter \glqq Albums\grqq , \glqq As rated by all RYM users \grqq  & \url{http://rateyourmusic.com/release/album/radiohead/ok\_}computer/                                        & OK Computer by Radiohead (Album, Alternative Rock): Reviews, Ratings, Credits, Song list - Rate Your Music                      & Album                                                   & \textasteriskcentered \textasteriskcentered \textasteriskcentered  \\
    ~                                    & ~                                                         & \url{http://rateyourmusic.com/release/album/my\_bloody\_valentine/loveless/}                                & Loveless by My Bloody Valentine (Album, Shoegaze): Reviews, Ratings, Credits, Song list - Rate Your Music                       & Album                                                   & \textasteriskcentered \textasteriskcentered \textasteriskcentered  \\
    ~                                    & ~                                                         & \url{http://rateyourmusic.com/release/album/neutral\_milk\_hotel/in\_the\_aeroplane\_over\_the\_sea/}       & In the Aeroplane Over the Sea by Neutral Milk Hotel (Album, Indie Folk): Reviews, Ratings, Credits, Song list - Rate Your Music & Album                                                   & \textasteriskcentered \textasteriskcentered \textasteriskcentered  \\   \hline
\end{longtable}
\end{landscape}
% Relevanz
% Qualit�t
% automatische Auswertbarkeit
%!TEX root = ../../Masterarbeit.tex
\begin{landscape}
\section*{Matrix der analysierten Suchdienste im Bereich Ereignisse}\label{app:events}
\begin{longtable}{|R{2.5cm}||R{2.5cm}|R{6.0cm}|R{4.0cm}|R{2.5cm}|l|}%{llllll}
\hline
    Zeitraum              & Suchbegriff                                                     & Link zum Suchergebnis                                                                     & �berschrift                                                                                                                                                                                                                & Beschreibung                                & Bewertung                                                          \\
\hline \hline
\endhead
    \textbf{google.de}        & ~                    & ~                                                                          & ~                                                                                                                               & ~                                                                     & ~         \\
    Achtziger        & \glqq events 80s\grqq          & \url{http://history1900s.about.com/od/timelines/tp/1980timeline.htm}             & 1980s Timeline                                                                                                                  & Liste von etwa 40 Ereignissen der 80er                                & \textasteriskcentered \textasteriskcentered          \\
    ~                & ~                    & \url{http://en.wikipedia.org/wiki/1980s}                                         & 1980s                                                                                                                           & Zusammenfassung der wichtigsten Ereignisse der 80er                   & \textasteriskcentered \textasteriskcentered \textasteriskcentered          \\
    ~                & ~                    & \url{http://www.only80s.com/80sLife-Timeline-Events.htm}                         & In the 1980s                                                                                                                    & Website �ber die 80er                                                 &          \\
    Neunziger        & \glqq events 90s\grqq          & \url{http://en.wikipedia.org/wiki/1990s}                                         & 1990s                                                                                                                           & Zusammenfassung der wichtigsten Ereignisse der 90er                   & \textasteriskcentered \textasteriskcentered \textasteriskcentered          \\
    ~                & ~                    & \url{http://en.wikipedia.org/wiki/Timeline\_of\_musical\_events}                 & Timeline of musical events                                                                                                      & �bersicht musikalischer Ereignisse der 90er                           & \textasteriskcentered          \\
    ~                & ~                    & \url{http://www.pureevents.ch/cm/}                                               & Pure 90\grq s                                                                                                                       & Event Page einer 90er-Party                                           & ~         \\
    Vietnamkrieg     & \glqq events vietnam war\grqq  & \url{http://history1900s.about.com/od/vietnamwar/a/vietnamtimeline.htm}          & Vietnam War Timeline                                                                                                            & Liste von Ereignissen des Vietnamkrieges                              & \textasteriskcentered \textasteriskcentered          \\
    ~                & ~                    & \url{http://vietnamwar.lib.umb.edu/chronology.html}                              & Major Events of the Vietnam War                                                                                                 & Liste von Ereignissen des Vietnamkrieges                              & \textasteriskcentered \textasteriskcentered          \\
    ~                & ~                    & \url{http://www.datesandevents.org/events-timelines/06-vietnam-war-timeline.htm} & Vietnam War Timeline                                                                                                            & Liste von Ereignissen des Vietnamkrieges                              & \textasteriskcentered \textasteriskcentered         \\
    ~                & ~                    & ~                                                                          & ~                                                                                                                               & ~                                                                     & ~         \\
    \textbf{en.wikipedia.org} & ~                    & ~                                                                          & ~                                                                                                                               & ~                                                                     & ~         \\
    Achtziger        & \glqq events 80s\grqq          & \url{http://en.wikipedia.org/wiki/Absolute\_80s}                                 & Absolute 80s                                                                                                                    & Artikel �ber Radiosender                                              & ~         \\
    ~                & ~                    & \url{http://en.wikipedia.org/wiki/Western\_Hockey\_League}                       & Western Hockey League                                                                                                           & Artikel �ber die Western Hockey League                                & ~         \\
    ~                & ~                    & \url{http://en.wikipedia.org/wiki/Maccabi\_Tel\_Aviv\_F.C.}                      & Maccabi Tel Aviv F.C.                                                                                                           & Artikel �ber einen Fu�ballclub                                        & ~         \\
    Neunziger        & \glqq events 90s\grqq          & \url{http://en.wikipedia.org/wiki/MFK\_Ko\%C5\%A1ice}                            & MFK Kosice                                                                                                                      & Artikel �ber einen Fu�ballclub                                        & ~         \\
    ~                & ~                    & \url{http://en.wikipedia.org/wiki/Australian\_Broadcasting\_Corporation}         & Australian Broadcasting Corporation                                                                                             & Artikel �ber einen Fernsehsender                                      & ~         \\
    ~                & ~                    & \url{http://en.wikipedia.org/wiki/Maccabi\_Tel\_Aviv\_F.C.}                      & Maccabi Tel Aviv F.C.                                                                                                           & Artikel �ber einen Fu�ballclub                                        & ~         \\
    Vietnamkrieg     & \glqq vietnam war events\grqq  & \url{http://en.wikipedia.org/wiki/Outline\_of\_the\_Vietnam\_War}                & Outline of the Vietnam War                                                                                                      & Sammlung von Eckdaten des Vietnamkrieges                              & \textasteriskcentered \textasteriskcentered          \\
    ~                & ~                    & \url{http://en.wikipedia.org/wiki/United\_States\_Marine\_Corps}                 & United States Marine Corps                                                                                                      & Artikel �ber die US-Marines                                           & ~         \\
    ~                & ~                    & \url{http://en.wikipedia.org/wiki/Vietnam}                                       & Vietnam                                                                                                                         & Artikel �ber Vietnam                                                  & ~         \\
    ~                & ~                    & ~                                                                          & ~                                                                                                                               & ~                                                                     & ~         \\
    \textbf{historyorb.com}   & google custom search & ~                                                                          & ~                                                                                                                               & ~                                                                     & ~         \\
    Achtziger        & \glqq 80s\grqq                 & \url{http://www.historyorb.com/europe/yugoslavia.php}                            & The Violent Breakup of Yugoslavia                                                                                               & Historischer Artikel                                                  & \textasteriskcentered \textasteriskcentered          \\
    ~                & ~                    & \url{http://www.historyorb.com/date/1980}                                        & Today in History for Year 1980                                                                                                  & Auflistung von Ereignissen im Jahre 1980                              & \textasteriskcentered          \\
    ~                & ~                    & \url{http://www.historyorb.com/date/1958/april/22}                               & Today in History for 22nd April 1958                                                                                            & Auflistung von Ereignissen am 22.04.1958                              & ~         \\
    Neunziger        & \glqq 90s\grqq                 & \url{http://www.historyorb.com/weddings/september/10}                            & Famous Weddings \& Divorces on 10th September                                                                                   & Auflistung wichtiger Eheschlie�ungen und Scheidungen am 10. September & ~         \\
    ~                & ~                    & \url{http://www.historyorb.com/weddings/september/8}                             & Famous Weddings \& Divorces on 8th September                                                                                    & Auflistung wichtiger Eheschlie�ungen und Scheidungen am 8. September  & ~         \\
    ~                & ~                    & \url{http://www.historyorb.com/date/1990}                                        & Today in History for Year 1990                                                                                                  & Auflistung von Ereignissen im Jahre 1990                              & \textasteriskcentered          \\
    Vietnamkrieg     & \glqq vietnam war\grqq         & \url{http://www.historyorb.com/war-history/vietnam-war}                          & Today in Vietnam War History                                                                                                    & Auflistung historischer Ereignisse w�hrend des Vietnamkrieges         & \textasteriskcentered \textasteriskcentered          \\
    ~                & ~                    & \url{http://www.historyorb.com/countries/vietnam}                                & Today in Vietnam History                                                                                                        & Auflistung historischer Ereignisse f�r Vietnam                        & \textasteriskcentered \textasteriskcentered          \\
    ~                & ~                    & \url{http://www.historyorb.com/countries/vietnam?p=2}                            & Today in Vietnam History (Part 2)                                                                                               & Auflistung historischer Ereignisse f�r Vietnam                        & \textasteriskcentered \textasteriskcentered         \\
    ~                & ~                    & ~                                                                          & ~                                                                                                                               & ~                                                                     & ~         \\
    \textbf{wolframalpha.com} & ~                    & Assumption                                                                 & ~                                                                                                                               & ~                                                                     & ~         \\
    Achtziger        & \glqq eighties\grqq            & Assuming \glqq eighties\grqq  is a word                                              & eighties  (English word)                                                                                                        & Input Interpretation & ~ \\
    ~                                              & ~                      & ~                                                      & 1 | noun | the decade from 1880 to 1889       & ~                    & ~ \\
    ~                                              & ~                      & ~                                                      & 2 | noun | the decade from 1980 to 1989       & ~                    & ~ \\
    ~                                              & ~                      & ~                                                      & 3 | noun | the time of life between 80 and 90 & Definitionen         & ~ \\
    ~                                              & ~                      & ~                                                      & eyteez  (IPA: \grq eitiz)            & englische Aussprache & ~ \\
    Neunziger        & \glqq nineties\grqq            & Assuming \glqq nineties\grqq  is a word                                              & nineties  (English word)                                                                                                        & Input Interpretation                                                  & ~         \\
    ~                         & ~                      & ~                                                      & 1 | noun | the decade from 1890 to 1899        & ~                    & ~ \\
    ~                         & ~                      & ~                                                      & 2 | noun | the decade from 1990 to 1999        & ~                    & ~ \\
    ~                         & ~                      & ~                                                      & 3 | noun | the time of life between 90 and 100 & Definitionen         & ~ \\
    ~                         & ~                      & ~                                                      & n\grq ahynteez  (IPA: n\grq aintiz) & englische Aussprache & ~ \\
    Vietnamkrieg     & \glqq vietnam war\grqq         & Assuming \glqq vietnam war\grqq  is a historical event                               & Vietnam War                                                                                                                     & Input Interpretation                                                  & \textasteriskcentered \textasteriskcentered \textasteriskcentered          \\
    ~                & ~                    & ~                                                                          & date, countries involved, people involved                                                                                       & Basic Information                                                     & \textasteriskcentered \textasteriskcentered \textasteriskcentered          \\
    ~                & ~                    & ~                                                                          & graph                                                                                                                           & Timeline                                                              & \textasteriskcentered \textasteriskcentered \textasteriskcentered          \\ \hline
\end{longtable}
\end{landscape}
% Relevanz
% Qualit�t
% automatische Auswertbarkeit

\chapter{Setup der Web-Applikation}
\label{app:setup}
\section{Entwicklungsumgebung}



\begin{table}[htdp]
	\begin{center}
		\begin{tabular}{|c|c|c|}
			\hline
			~ & \textbf{Tool} & \textbf{Version}\\
			\hline
			\hline
			Hardware & MacBook Pro 2009 mid & ~ \\	
				& Prozessor & 2,26 GHz Intel Core 2 Duo\\
				& Speicher & 8 GB 1067 MHz DDR3\\
				& Grafikkarte & NVIDIA GeForce 9400M 256 MB\\
				& Software & OS X 10.9.2 (13C64)\\
			\hline
			Software & Node.js & Version v0.10.26 \\
				& MongoDB & Shell Version: 2.4.9 \\
				& ~ & DB Version: v2.4.9 \\
			\hline
			Module & express & ~3.3.4 \\
				& jade & ~0.34.0 \\
				& mongoose & ~3.6.14 \\
				& grunt & ~0.4.1 \\
				& grunt-develop &  ~0.2.2 \\ 
				& grunt-contrib-watch &  ~0.5.3 \\ 
				& request &  ~2.27.0 \\ 
				& time-grunt &  ~0.1.1 \\ 
				& load-grunt-tasks &  ~0.2.0 \\ 
				& grunt-contrib-copy &  ~0.5.0 \\ 
				& stylus &  ~0.42.0 \\ 
				& grunt-contrib-stylus &  ~0.12.0 \\ 
				& nib &  ~1.0.2 \\ 
				& grunt-contrib-clean &  ~0.5.0 \\ 
				& node-rest-client &  ~0.7.5 \\ 
				& async &  ~0.2.10 \\ 
				& coverart &  0.0.1 \\ 
				& cheerio &  ~0.13.1 \\ 
				& imdb-api &  ~1.3.3 \\ 
				& node-fs &  ~0.1.7 \\ 
				& grunt-bower &  ~0.8.4 \\ 
				& tomatoes &  0.0.1" \\
			\hline
			Browser (Tested) & Google Chrome & Version 33.0.1750.152 \\
				& Safari & Version 7.0.2 (9537.74.9) \\
				& Mozilla Firefox & Version 26.0 \\
			\hline
		\end{tabular}
	\end{center}
	\caption{Verwendete Hard- und Software}
	\label{tab:hardSoftware}
\end{table}%


\section{Vorraussetzungen und lokale Installation}
Zum lokalen Aufsetzen der Web-Applikation muss Node.js sowie MongoDB installiert sein.\footnote{Die genauen Versionen sind der Tabelle \ref{tab:hardSoftware} zu entnehmen und werden hier nicht erneut aufgef�hrt.} �ber die Konsole(Windows) oder im Terminal(Mac OS X) in den Ordner \glqq CultureChronicles\grqq \ navigieren und \lstinline$npm install$ ausf�hren um alle ben�tigten Module zu installieren. Zur Ausf�hrung von \arbeitstitel \ muss MongoDB laufen.\footnote{Auf Mac OS X wird MongoDB �ber den Befehl \lstinline$mongod$ im Terminal gestartet.} Nun kann die Web-Applikation mit dem Befehl \lstinline$grunt$ gestartet werden und ist �ber \url{localhost:3000/} zu erreichen.