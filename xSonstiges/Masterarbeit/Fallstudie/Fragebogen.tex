
% Masterarbeit
\documentclass[10pt, a4paper,onecolumn]{article}
% Optionale Parameter:
% Schriftgröße: 10pt, 11pt, 12pt - Standard ist 10pt
% Papierformat: letterpaper, legalpaper, a4paper, a5paper
% Spalten: onecolumn, twocolumn
% Auch als Kommando: onecolumn{}, twocolumn{}

% Präamble - Setup
\usepackage[utf8]{inputenc}
\usepackage[T1]{fontenc}

% Worttrennung auf deutsch umstellen
\usepackage[ngerman]{babel}

% Monospace Schrift
%\usepackage{courier} % Courier

% Schriftart Charter von Matthew Carter
\usepackage{charter}

% Zeilenabstand, macht das setstretching Kommando verfügbar
\usepackage{setspace}
% (Schriftgröße 10pt + Durchschuss 3pt) / Schriftgröße 10pt ->
\setstretch{1.3}
% Grafik - Optionaler Parameter für den Grafiktreiber pdftex
\usepackage[pdftex]{graphicx}

% Kopf- Fußzeile änder n- Erst NACH geometry definieren
%\usepackage{fancyhdr}
%\pagestyle{fancyplain}
% Leert den Header
%\fancyhead{}
% In der Präambel für das gesamte Dokument
% Später im Dokument ändert das die Kopf- bzw. Fußzeile für das restliche Dokument
%\lhead{\leftmark{}}%\textbf{LINKS}}
%\chead{}%MITTE}
%\rhead{\rightmark{}}%RECHTS\rule{3pt}{3pt}}
%
%\lfoot{\textbf{LINKS}}
%\cfoot{MITTE}
%\rfoot{\thepage{}}

% Label-Beschriftung, kleine Schriftgröße, Bezeichner fett und Text kursiv
\usepackage[labelsep=colon, font=small, labelfont=bf, textfont=it]{caption}
% Farbe
\usepackage[pdftex]{color}
% Farbe definieren: {Name}{Farbmodell}{Werte}
% RGB 0 99 99
% HTW grün:
\definecolor{Schmuckfarbe_1}{rgb}{0.32, 0.73, 0}
% Dunkeles Grau
\definecolor{Schmuckfarbe_2}{gray}{0.4}

% Überschriften anpassen
\usepackage{sectsty}
%
%\allsectionsfont{\sffamily{}}
\sectionfont{\color{Schmuckfarbe_1}}
\subsectionfont{\color{Schmuckfarbe_1}}

% Optischer Randausgleich
\usepackage[activate]{pdfcprot}
% Sperrung von Versalien
\usepackage{soul}
% Vernünftige URL Trennung und highlighting
\usepackage[hyphens]{url}

\usepackage{latexsym}

% Europäische Abstände
\frenchspacing{}
% plain, headings, empty
\pagestyle{empty}
% ENDE Präambel
%
\begin{document}

\subsection*{Test-Fragebogen Zauberwald}
\begin{table}[h]
\setstretch{1.3}
\begin{tabular}{ll}
Datum: &\\
\\
Startuhrzeit: & \\
Enduhrzeit: & \\
Spieldauer: & \\
\end{tabular}
\end{table}

\subsection*{Angaben zur Person}

\begin{table}[h]
\setstretch{1.3}
\begin{tabular}{llll}
Name:         &  &  & \\
\\
Alter:        &  &  & \\
\\
Sehfähigkeit: & $\Box{}$ blind           & Geschlecht: & $\Box{}$ weiblich\\
			  & $\Box{}$ sehgeschädigt   & 			   & $\Box{}$ männlich\\
			  & $\Box{}$ uneingeschränkt & 			   & $\Box{}$ unbestimmt\\
\end{tabular}
\end{table}



\subsection*{Fragen zum Testspiel}

\subsubsection*{Wie gut konntest du dich im Spiel orientieren?}
\noindent
sehr gut $\Box{}$\hspace{0.5cm}gut $\Box{}$\hspace{0.5cm}mittelmäßig
$\Box{}$\hspace{0.5cm}schlecht $\Box{}$\hspace{0.5cm}gar nicht $\Box{}$

\subsubsection*{Wie gut bist du mit der Steuerung zurecht gekommen?}
\noindent
sehr gut $\Box{}$\hspace{0.5cm}gut $\Box{}$\hspace{0.5cm}mittelmäßig
$\Box{}$\hspace{0.5cm}schlecht $\Box{}$\hspace{0.5cm}gar nicht $\Box{}$

\subsubsection*{Hat dir das Spiel Spaß gemacht?}
\noindent
ja $\Box{}$\hspace{0.5cm}eher ja $\Box{}$\hspace{0.5cm}neutral
$\Box{}$\hspace{0.5cm}eher nicht $\Box{}$\hspace{0.5cm}nein $\Box{}$

\subsubsection*{Wie würdest du den Schwierigkeitsgrad einschätzen?}
\noindent
zu leicht $\Box{}$\hspace{0.5cm}leicht $\Box{}$\hspace{0.5cm}anspruchsvoll
$\Box{}$\hspace{0.5cm}schwer $\Box{}$\hspace{0.5cm}zu schwer $\Box{}$

\newpage
\subsection*{Fragen zum Nutzungsverhalten}

\subsubsection*{Wie oft benutzt du einen Computer?}
\noindent
selten $\Box{}$\hspace{0.5cm}mehrmals im Monat $\Box{}$\hspace{0.5cm}
mehrmals in der Woche $\Box{}$\hspace{0.5cm}täglich $\Box{}$

\subsubsection*{Wie oft spielst du Computerspiele?}
\noindent
selten $\Box{}$\hspace{0.5cm}mehrmals im Monat $\Box{}$\hspace{0.5cm}
mehrmals in der Woche $\Box{}$\hspace{0.5cm}täglich $\Box{}$

\subsubsection*{Wie würdest du dein Interesse an Computerspielen beschreiben?}\noindent
sehr hoch $\Box{}$\hspace{0.5cm}hoch $\Box{}$\hspace{0.5cm}gering
$\Box{}$\hspace{0.5cm}kein Interesse $\Box{}$

\subsection*{Für blinde Menschen}
\noindent
Welche Ein- und Ausgabegeräte benutzt du in Verbindung mit einem Computer?
\begin{table}[h]
\setstretch{1.3}
\begin{tabular}{p{\textwidth}}
\\
\hline
\\
\hline
\\
\hline
\\
\hline
\\
\hline
\end{tabular}
\end{table}


\subsection*{Anmerkungen}
\begin{table}[h]
\setstretch{1.3}
\begin{tabular}{p{\textwidth}}
\\
\hline
\\
\hline
\\
\hline
\\
\hline
\\
\hline
\end{tabular}
\end{table}

\newpage
\subsection*{Informationen zum Spiel}
\subsubsection*{Audioclient}
\begin{itemize}
\item{}Menüsteuerung Listen
	\begin{itemize}
	\item{}Pfeiltaste hoch: Einen Eintrag nach oben gehen
	\item{}Pfeiltaste runter: Einen Eintrag nach unten gehen
	\item{}Pfeiltaste rechts: Einen Eintrag auswählen
	\item{}Pfeiltaste links: Das Menü verlassen
	\end{itemize}
\item{}Bewegung im Raum über wasd-Tasten
	\begin{itemize}
	\item{}w: Spielfigur geht ein Feld nach oben
	\item{}a: Spielfigur geht ein Feld nach links
	\item{}s: Spielfigur geht ein Feld nach unten
	\item{}d: Spielfigur geht ein Feld nach rechts
	\end{itemize}
\item{}Menüs im Spiel
	\begin{itemize}
	\item{}q: Inventar öffnen
	\item{}e: Interaktionsmenü öffnen
	\item{}x: Position im Raum abfragen
	\item{}y: Alle Gegenstände des Raums abfragen
	\end{itemize}
\end{itemize}

\subsubsection*{Pygameclient}
\begin{itemize}
\item{}Maussteuerung
	\begin{itemize}
	\item{}Links-Klick auf eine Position: Spielfigur bewegt sich dort hin
	\item{}Rechts-Klick auf ein Objekt öffnet visuelles Interaktionsmenü
	\item{}Drag\&Drop Gegenstände können ins Inventar oder auf andere Objekte gezogen werden
	\end{itemize}
\end{itemize}

\end{document}